
\documentclass[11pt]{article} % use larger type; default would be 10pt
\usepackage{framed}
\usepackage[utf8]{inputenc} % set input encoding (not needed with XeLaTeX)
\usepackage{geometry} % to change the page dimensions
\geometry{a4paper} % or letterpaper (US) or a5paper or....


\usepackage{graphicx} % support the \includegraphics command and options

% \usepackage[parfill]{parskip} % Activate to begin paragraphs with an empty line rather than an indent

%%% PACKAGES
\usepackage{booktabs} % for much better looking tables
\usepackage{array} % for better arrays (eg matrices) in maths
\usepackage{paralist} % very flexible & customisable lists (eg. enumerate/itemize, etc.)
\usepackage{verbatim} % adds environment for commenting out blocks of text & for better verbatim
\usepackage{subfig} % make it possible to include more than one captioned figure/table in a single float
% These packages are all incorporated in the memoir class to one degree or another...
\usepackage{framed}

%%% HEADERS & FOOTERS
\usepackage{fancyhdr} % This should be set AFTER setting up the page geometry
\pagestyle{fancy} % options: empty , plain , fancy
\renewcommand{\headrulewidth}{0pt} % customise the layout...
\lhead{}\chead{}\rhead{}
\lfoot{}\cfoot{\thepage}\rfoot{}

%%% SECTION TITLE APPEARANCE
\usepackage{sectsty}
\allsectionsfont{\sffamily\mdseries\upshape} % (See the fntguide.pdf for font help)
% (This matches ConTeXt defaults)

%%% ToC (table of contents) APPEARANCE
\usepackage[nottoc,notlof,notlot]{tocbibind} % Put the bibliography in the ToC
\usepackage[titles,subfigure]{tocloft} % Alter the style of the Table of Contents
\renewcommand{\cftsecfont}{\rmfamily\mdseries\upshape}
\renewcommand{\cftsecpagefont}{\rmfamily\mdseries\upshape} % No bold!

\begin{document}

Course Syllabus

\begin{description}
\item [Week 1:] What are networks and what use is it to study them?
\begin{itemize}
\item\item[Concepts:] nodes, edges, adjacency matrix, one and two-mode networks, node degree
\item[Activity:] Upload a social network (e.g. your Facebook social network into Gephi and visualize it ).
\end{itemize}

\item [Week 2:] Random network models:] Erdos-Renyi and Barabasi-Albert
\begin{itemize}\item[Concepts:] connected components, giant component, average shortest path, diameter, breadth-first search, preferential attachment
\item[Activity:] Create random networks, calculate component distribution, average shortest path, evaluate impact of structure on ability of information to diffuse
\end{itemize}

\item [Week 3:] Network centrality
\begin{itemize}\item[Concepts:] betweenness, closeness, eigenvector centrality (+ PageRank), network centralization
\item[Activity:] calculate and interpret node centrality for real-world networks (your Facebook graph, the Enron corporate email network, Twitter networks, etc.)
\end{itemize}

\item [Week 4:] Community
\begin{itemize}\item[Concepts:] clustering, community structure, modularity, overlapping communities
Activities:] detect and interpret disjoint and overlapping communities in a variety of networks (scientific collaborations, political blogs, cooking ingredients, etc.)
\end{itemize}

\item [Week 5:] Small world network models, optimization, strategic network formation and search
\begin{itemize}\item[Concepts:] small worlds, geographic networks, decentralized search
\item[Activity:] Evaluate whether several real-world networks exhibit small world properties, simulate decentralized search on different topologies, evaluate effect of small-world topology on information diffusion.
\end{itemize}



\item [Week 6:] Contagion, opinion formation, coordination and cooperation
\begin{itemize}\item[Concepts:] simple contagion, threshold models, opinion formation
\item[Activity:] Evaluate via simulation the impact of network structure on the above processes
\end{itemize}



\item [Week 7:] Cool and unusual applications of SNA

\begin{itemize}
\item Hidalgo et al. Predicting economic development using product space networks (which countries produce which products)
\item Ahn et al., and Teng et al.
 Learning about cooking from ingredient and flavor networks
\item Lusseau et al. Social networks of dolphins
others TBD
\end{itemize}
\begin{itemize}
\item[Activity:] hands-on exploration of these networks using concepts learned earlier in the course
\end{itemize}

\item [Week 8:] SNA and online social networks
\begin{itemize}\item[Concepts:] how services such as Facebook, LinkedIn, Twitter, CouchSurfing, etc. are using SNA to understand their users and improve their functionality
\item[Activity:] read recent research by and based on these services and 
learn how SNA concepts were applied
\end{itemize}


\end{description}

\end{document}