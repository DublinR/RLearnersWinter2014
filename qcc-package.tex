\documentclass[12pt]{article}
\usepackage{amsmath}
\usepackage{amssymb}
\usepackage{framed}
\usepackage{graphicx}

\begin{document}

\section*{\texttt{qcc}: Quality Control Charts}
\textit{(yhat article by Drew Conway)}\\
\textbf{qcc} is a library for statistical quality control. Back in the 1950s, the now defunct Western Electric Company was looking for a better way to detect problems with telephone and eletrical lines. They came up with a set of rules to help them identify problematic lines. \\

\noindent The rules look at the historical mean of a series of datapoints and based on the standard deviation, the rules help judge whether a new set of points is experiencing a mean shift.\\

\noindent The classic example is monitoring a machine that produces lug nuts. Let's say the machine is supposed to produce 2.5 inch long lug nuts. We measure a series of lug nuts: 2.48, 2.47, 2.51, 2.52, 2.54, 2.42, 2.52, 2.58, 2.51. Is the machine broken? Well it's hard to tell, but the Western Electric Rules can help.
\begin{framed}
\begin{verbatim}
library(qcc)
 
# series of value w/ mean of 10 
# with a little random noise added in
x <- rep(10, 100) + rnorm(100)

# a test series w/ a mean of 11
new.x <- rep(11, 15) + rnorm(15)

# qcc will flag the new points
qcc(x, newdata=new.x, type="xbar.one")

\end{verbatim}
\end{framed}
While you might not be monitoring telephone lines, qcc can help you monitor transaction volumes, visitors or logins on your website, database operations, and lots of other processes.

%----------------------------------------------------- %
\newpage
\begin{itemize}
\item Shewhart quality control charts for continuous, attribute and count data. 
\item Cusum and EWMA charts. 
\item Operating characteristic curves. 
\item Process capability analysis. 
\item Pareto chart and cause-and-effect chart. 
\item Multivariate control charts.
\end{itemize}
\end{document}
