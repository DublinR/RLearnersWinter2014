
Performing a principal component analysis in R 
%======================================================================================================================%


Principal component analysis
 
A principal component analysis (or PCA) is a way of simplifying a complex multivariate dataset. It helps to expose the underlying sources of variation in the data.
 
You can perform a principal component analysis with the princomp function as shown below. 
> princomp(dataset) 
The dataset should contain numeric variables only. If there are any non-numeric variables in your dataset, you must exclude them with bracket notation or with the subset function.
 
The princomp output displays the standard deviations of the components. However there are more elements of the output that are not automatically displayed, including the loadings and scores. You can save the all of this output to an object, as shown below.
 > modelname<-princomp(dataset) 
Once you have saved the output to an object, you can use further functions to view the various elements of the output. For example, you can use the summary function to view the proportion of the total variance explained by each component:
 > summary(modelname) 
To view the loadings for each component, use the command:
 > modelname$loadings 
Similarly you can view the scores for each of the observations as shown:
 > modelname$scores 
To create a scree plot, please see the article Creating a scree plot with R.
 
Example: Principal component analysis using the iris data
 
Consider the iris dataset (included with R) which gives the petal width, petal length, sepal width, sepal length and species for 150 irises. To view more information about the dataset, enter help(iris).
 
You can view the dataset by entering the dataset name:
 > iris     Sepal.Length Sepal.Width Petal.Length Petal.Width    Species
1            5.1         3.5          1.4         0.2     setosa
2            4.9         3.0          1.4         0.2     setosa
3            4.7         3.2          1.3         0.2     setosa
4            4.6         3.1          1.5         0.2     setosa
5            5.0         3.6          1.4         0.2     setosa
...
150          5.9         3.0          5.1         1.8  virginica 
The dataset contains a factor variable (Species) which must be excluded when performing the PCA. So to perform the analysis and save the results to an object, use the command:
 > irispca<-princomp(iris[-5]) 
To view the proportion of the total variance explained by each component, use the command:
 > summary(irispca) Importance of components:
                          Comp.1     Comp.2     Comp.3      Comp.4
Standard deviation     2.0494032 0.49097143 0.27872586 0.153870700
Proportion of Variance 0.9246187 0.05306648 0.01710261 0.005212184
Cumulative Proportion  0.9246187 0.97768521 0.99478782 1.000000000 
From the output we can see that 92.4% of the variation in the dataset is explained by the first component alone, and 97.8% is explained by the first two components.
 
To view the loadings for the components, use the command:
 > irispca$loadings Loadings:
             Comp.1 Comp.2 Comp.3 Comp.4
Sepal.Length  0.361 -0.657  0.582  0.315
Sepal.Width         -0.730 -0.598 -0.320
Petal.Length  0.857  0.173        -0.480
Petal.Width   0.358        -0.546  0.754

               Comp.1 Comp.2 Comp.3 Comp.4
SS loadings      1.00   1.00   1.00   1.00
Proportion Var   0.25   0.25   0.25   0.25
Cumulative Var   0.25   0.50   0.75   1.00 
To view the scores for each observation, use the command:
 > irispca$scores              Comp.1       Comp.2       Comp.3        Comp.4
  [1,] -2.684125626 -0.319397247  0.027914828  0.0022624371
  [2,] -2.714141687  0.177001225  0.210464272  0.0990265503
  [3,] -2.888990569  0.144949426 -0.017900256  0.0199683897
  [4,] -2.745342856  0.318298979 -0.031559374 -0.0755758166
  [5,] -2.728716537 -0.326754513 -0.090079241 -0.0612585926
  [6,] -2.280859633 -0.741330449 -0.168677658 -0.0242008576
  [7,] -2.820537751  0.089461385 -0.257892158 -0.0481431065
  [8,] -2.626144973 -0.163384960  0.021879318 -0.0452978706
  [9,] -2.886382732  0.578311754 -0.020759570 -0.0267447358
 [10,] -2.672755798  0.113774246  0.197632725 -0.0562954013
...
[150,]  1.390188862  0.282660938 -0.362909648 -0.1550386282 
This example is continued in the article Creating a scree plot with R.


%==========================================================================%


Creating a scree plot in R 






Scree plot
 
A scree plot displays the proportion of the total variation in a dataset that is explained by each of the components in a principle component analysis. It helps you to identify how many of the components are needed to summarise the data.
 
To create a scree plot of the components, use the screeplot function.
 > screeplot(modelname) 
where modelname is the name of a previously saved principle component analysis, created with the princomp function as explained in the article Performing a principle component analysis in R.
 
Example: Scree plot for the iris dataset
 
In a the article Performing a principal component analysis with R we performed a principle component analysis for the iris dataset, and saved it to an object named irispca.
 
To create a scree plot of the components, use the command:
 > screeplot(irispca) 
The result is shown below.
 


From the scree plot we can see that the amount of variation explained drops dramatically after the first component. This suggests that just one component may be sufficient to summarise the data.
