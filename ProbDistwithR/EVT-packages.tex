\begin{frame}
The Extreme Value Theory (EVT) is extensively used for modelling very large and/or very small events. 
Usually the focus of the analysis is the estimation of very extreme quantiles or tail probabilities. 
It is widely used in several areas, such as environment, insurance, whether and hydrology. 
Several R packages have been developed for fitting models in this framework. 
%This course aims to present the most common statistical tools of the EVT, as well as teaching the use of a R package.
\end{frame}
\begin{frame}
BSquare

BSquare is a package that models the quantile function using splines for non extremes and the generalized Pareto distribution for extremes. For more information, see its web page.

copula

copula is a package containing functions for exploring and modeling several commonly used copulas. MLE, pseudo-MLE and method of moments are all avialable. It does not allow for non-stationary regression, but does allow for multivariate modeling (as you would expect).

evd

evd (extreme value distributions) is and add-on package for the R system. It extends simulation, distribution, quantile and density functions to univariate and multivariate parametric extreme value distributions, and provides fitting functions which calculate maximum likelihood estimates for univariate and bivariate models, and for univariate and bivariate threshold models.

evdbayes

evdbayes is and add-on package for the R system. It provides functions for the bayesian analysis of extreme value models, using MCMC methods.
\end{frame}
\begin{frame}
evir

evir is and add-on package for the R system. It is an R port (conversion) of Version 3 of Alexander McNeil's S library EVIS (Extreme Values in S). It contains functions for extreme value theory, which may be divided into the following groups; exploratory data analysis, block maxima, peaks over thresholds (univariate and bivariate), point processes, gev/gpd distributions.

ismev

ismev is and add-on package for the R system. It is an R port of S functions written by Stuart Coles to support (univariate) extreme value modelling (or modeling, if you're on the other side of the pond), including the computations carried out in Coles (2001). The functions may be divided into the following groups; maxima/minima, order statistics, peaks over thresholds and point processes. Coles (2001) is a textbook that provides an introduction to the topic at a relatively simple statistical level.

Versions 1.37 and 1.38 include only very minor changes. The first was an update to add a NAMESPACE for consistency and usability with the newer versions of R. The latter fixed a couple of unimportant warnings: (i) a warning was given in the calls to gev.diag and gpd.diag because of the prob = TRUE argument that was not used, (ii) a warning about a partial argument match in one of the functions (this was potentially more serious in the event of change that does not allow partial argument matching).

Version 1.36 has a minor change to allow users to apply optional arguments to gpd.fit and pp.fit when calling gpd.fitrange and pp.fitrange, respectively.

Changes in version 1.34 have fixed a bug in the numerical derivative used by gpd.rl by using the exact gradients instead of the numerical ones. The change has also been made to gev.rl to use the exact gradient.

Changes in version 1.33 include the ability for the user to provide their own initial values for the MLE numerical optimization routine. Also, classes are assigned to some of the fits. Though ismev does not (presently) use the classes, extRemes does make use of them. In the future, ismev may as well, but for now there is a desire to keep this pacakge as similar as possible to the original that accompanied Coles' book.

New to version 1.32 are a couple of bug fixes. The first invovles fitting point process models with parameter covariates. A more accurate approximation to the exceedance rate intensity is now used at the cost of computational efficiency. Second, the qq-plot for the point process with covariates has also been fixed, and should now be correct.
\end{frame}
\begin{frame}

extRemes

extRemes is and add-on package for the R system, written by Eric Gilleland, Rick Katz and Greg Young, and maintained by Eric Gilleland. It provides a windows GUI for the ismev package, allowing easy use of the tools that the ismev package provides, in addition to a few extra useful functions. See the extRemes web site for details on the package and version updates.

extremevalues

Software package for detecting extreme values in one-dimensional data. According to the description section of the help file for the package (version 2.1), the package implements outlier detection methods introduced in a discussion paper by the package author, Mark van der Loo (M.P.J. van der Loo, Discussion paper 10003, Statistics Netherlands, The Hague, 2010; available at http://www.cbs.nl).

fExtremes

fExtremes is and add-on package for the R system, maintained and primarily written by Diethelm Wuertz. The package contains functions for the exploratory data analysis of extreme values for insurance, economic and financial applications. It also brings together many of the elements of the packages evd, evir and ismev. The fExtremes package comprises part of the Rmetrics software collection. See the Rmetrics web site for details.

lmom

Functions related to L-moments, includes functions to compute L-moment estimates for extreme value distribution parameters.

\end{frame}
\begin{frame}
lmomRFA

Package written and maintained by J.R.M. Hosking containing functions for regional frequency analysis (RFAA) using the methods of Hosking and Wallis (1997) Regional frequency analysis: an approach based on L-moments, Cambridge University Press (newest edition, 2005; 244 pp., ISBN-10: 0521019400).

lmomco

Package written by William H. Asquith to compute L-moments, trimmed L-momes, L-comoments, and probability-weighted moment estimation for many distributions including extreme value distributions.

POT

Functions written by Mathieu Ribatet for performing peak over threshold (POT) analysis for both univariate and bivariate cases.

The SpatialExtremes Package

A new package by Mathieu Ribatet for performing multivariate and other spatial extremes methods, called SpatialExtremes, is now available via CRAN. It is still in a development stage, but check its web site for further information at http://spatialextremes.r-forge.r-project.org/

texmex

A relatively new package with functions for performing the conditional EVA approach introduced in Heffernan and Tawn (2004), J. R. Statist. Soc. B, 66 (3), 497 - 546. Also contains good functions for fitting the GP df to data.
\end{frame}
\begin{frame}
VGAM

VGAM is a package for fitting vector generalized additive models. That is, it allows for modeling parameters as linear or smooth functions of covariates.
\end{frame}
