\documentclass[main.tex]{subfiles}
\begin{document}
\section{``Orthodont Data Set" -  Example (nlme) }

\begin{itemize}
\item Investigators at the University of North Carolina Dental School followed the growth of 27 children (16 males, 11 females) from age 8 until age 14. 
\item Every two years they measured the distance between the pituitary and the pterygomaxillary fissure, two points that are easily identified on x-ray exposures of the side of the head.
\end{itemize}

\begin{framed}
\begin{verbatim}
library(nlme)

data(Orthodont)

summary(Orthodont)

\end{verbatim}
\end{framed}

\begin{verbatim}
> summary(Orthodont)
    distance          age          Subject       Sex    
 Min.   :16.50   Min.   : 8.0   M16    : 4   Male  :64  
 1st Qu.:22.00   1st Qu.: 9.5   M05    : 4   Female:44  
 Median :23.75   Median :11.0   M02    : 4              
 Mean   :24.02   Mean   :11.0   M11    : 4              
 3rd Qu.:26.00   3rd Qu.:12.5   M07    : 4              
 Max.   :31.50   Max.   :14.0   M08    : 4              
                                (Other):84
\end{verbatim}


\end{document}