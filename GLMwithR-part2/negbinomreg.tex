
R Data Analysis Examples: Negative Binomial Regression

Negative binomial regression is for modeling count variables, usually for over-dispersed count outcome variables.

This page uses the following packages. Make sure that you can load them before trying to run the examples on this page. If you do not have a package installed, run: install.packages("packagename"), or if you see the version is out of date, run: update.packages().

\begin{framed}
\begin{verbatim}

\end{verbatim}
require(foreign)
require(ggplot2)
require(MASS)
\end{framed}
%Version info: Code for this page was tested in R Under development (unstable) (2013-01-06 r61571)
%On: 2013-01-22
%With: MASS 7.3-22; ggplot2 0.9.3; foreign 0.8-52; knitr 1.0.5 

Please note: The purpose of this page is to show how to use various data analysis commands. It does not cover all aspects of the research process which researchers are expected to do. In particular, it does not cover data cleaning and checking, verification of assumptions, model diagnostics or potential follow-up analyses.

\subsection*{Examples of negative binomial regression}

\begin{description}
\item[Example 1.] School administrators study the attendance behavior of high school juniors at two schools. Predictors of the number of days of absence include the type of program in which the student is enrolled and a standardized test in math. 

\item[Example 2.] A health-related researcher is studying the number of hospital visits in past 12 months by senior citizens in a community based on the characteristics of the individuals and the types of health plans under which each one is covered.

\end{description}
%========================================================= %
\subsection*{Description of the data}

Let's pursue Example 1 from above.

We have attendance data on 314 high school juniors from two urban high schools in the file nb_data. The response variable of interest is days absent, daysabs. The variable math gives the standardized math score for each student. The variable prog is a three-level nominal variable indicating the type of instructional program in which the student is enrolled.

Let's look at the data. It is always a good idea to start with descriptive statistics and plots.

\begin{framed}
\begin{verbatim}

dat <- read.dta("http://www.ats.ucla.edu/stat/stata/dae/nb_data.dta")
dat <- within(dat, {
	prog <- factor(prog, levels = 1:3, labels = c("General", "Academic", "Vocational"))
	id <- factor(id)
})
\end{verbatim}
\end{framed}
\begin{verbatim}
summary(dat)

##        id         gender         math         daysabs     
##  1001   :  1   female:160   Min.   : 1.0   Min.   : 0.00  
##  1002   :  1   male  :154   1st Qu.:28.0   1st Qu.: 1.00  
##  1003   :  1                Median :48.0   Median : 4.00  
##  1004   :  1                Mean   :48.3   Mean   : 5.96  
##  1005   :  1                3rd Qu.:70.0   3rd Qu.: 8.00  
##  1006   :  1                Max.   :99.0   Max.   :35.00  
##  (Other):308                                              
##          prog    
##  General   : 40  
##  Academic  :167  
##  Vocational:107  
##                  
##                  
##                  
## 

ggplot(dat, aes(daysabs, fill = prog)) 
  + geom_histogram(binwidth = 1) 
  + facet_grid(prog ~ ., 
       margins = TRUE, 
       scales = "free")

\end{verbatim}



Each variable has 314 valid observations and their distributions seem quite reasonable. The unconditional mean of our outcome variable is much lower than its variance. 

Let's continue with our description of the variables in this dataset. The table below shows the average numbers of days absent by program type and seems to suggest that program type is a good candidate for predicting the number of days absent, our outcome variable, because the mean value of the outcome appears to vary by prog. The variances within each level of prog are higher than the means within each level. These are the conditional means and variances. These differences suggest that over-dispersion is present and that a Negative Binomial model would be appropriate. 



with(dat, tapply(daysabs, prog, function(x) {
	sprintf("M (SD) = %1.2f (%1.2f)", mean(x), sd(x))
}))

##                 General                Academic              Vocational 
## "M (SD) = 10.65 (8.20)"  "M (SD) = 6.93 (7.45)"  "M (SD) = 2.67 (3.73)"

Analysis methods you might consider

Below is a list of some analysis methods you may have encountered. Some of the methods listed are quite reasonable, while others have either fallen out of favor or have limitations. 
\begin{itemize}
\item \textbf{Negative binomial regression} -Negative binomial regression can be used for over-dispersed count data, that is when the conditional variance exceeds the conditional mean. It can be considered as a generalization of Poisson regression since it has the same mean structure as Poisson regression and it has an extra parameter to model the over-dispersion. If the conditional distribution of the outcome variable is over-dispersed, the confidence intervals for the Negative binomial regression are likely to be narrower as compared to those from a Poisson regression model.
\item \textbf{Poisson regression }- Poisson regression is often used for modeling count data. Poisson regression has a number of extensions useful for count models. 
\item \textbf{Zero-inflated regression model} - Zero-inflated models attempt to account for excess zeros. In other words, two kinds of zeros are thought to exist in the data, "true zeros" and "excess zeros". Zero-inflated models estimate two equations simultaneously, one for the count model and one for the excess zeros.
\item \textbf{OLS regression }- Count outcome variables are sometimes log-transformed and analyzed using OLS regression. Many issues arise with this approach, including loss of data due to undefined values generated by taking the log of zero (which is undefined), as well as the lack of capacity to model the dispersion.
\end{itemize}
%======================================================== %
Negative binomial regression analysis

Below we use the glm.nb function from the MASS package to estimate a negative binomial regression.



summary(m1 <- glm.nb(daysabs ~ math + prog, data = dat))

## 
## Call:
## glm.nb(formula = daysabs ~ math + prog, data = dat, init.theta = 1.032713156, 
##     link = log)
## 
## Deviance Residuals: 
##    Min      1Q  Median      3Q     Max  
## -2.155  -1.019  -0.369   0.229   2.527  
## 
## Coefficients:
##                Estimate Std. Error z value Pr(>|z|)    
## (Intercept)     2.61527    0.19746   13.24  < 2e-16 ***
## math           -0.00599    0.00251   -2.39    0.017 *  
## progAcademic   -0.44076    0.18261   -2.41    0.016 *  
## progVocational -1.27865    0.20072   -6.37  1.9e-10 ***
## ---
## Signif. codes:  0 '***' 0.001 '**' 0.01 '*' 0.05 '.' 0.1 ' ' 1
## 
## (Dispersion parameter for Negative Binomial(1.033) family taken to be 1)
## 
##     Null deviance: 427.54  on 313  degrees of freedom
## Residual deviance: 358.52  on 310  degrees of freedom
## AIC: 1741
## 
## Number of Fisher Scoring iterations: 1
## 
## 
##               Theta:  1.033 
##           Std. Err.:  0.106 
## 
##  2 x log-likelihood:  -1731.258
•R first displays the call and the deviance residuals. Next, we see the regression coefficients for each of the variables, along with standard errors, z-scores, and p-values. The variable math has a coefficient of -0.006, which is statistically significant. This means that for each one-unit increase in math, the expected log count of the number of days absent decreases by 0.006. The indicator variable shown as progAcademic is the expected difference in log count between group 2 and the reference group (prog=1). The expected log count for level 2 of prog is 0.44 lower than the expected log count for level 1. The indicator variable for progVocational is the expected difference in log count between group 3 and the reference group.The expected log count for level 3 of prog is 1.28 lower than the expected log count for level 1. To determine if prog itself, overall, is statistically significant, we can compare a model with and without prog. The reason it is important to fit separate models, is that unless we do, the overdispersion parameter is held constant.



m2 <- update(m1, . ~ . - prog)
anova(m1, m2)

## Likelihood ratio tests of Negative Binomial Models
## 
## Response: daysabs
##         Model  theta Resid. df    2 x log-lik.   Test    df LR stat.
## 1        math 0.8559       312           -1776                      
## 2 math + prog 1.0327       310           -1731 1 vs 2     2    45.05
##     Pr(Chi)
## 1          
## 2 1.652e-10

\begin{itemize}
\item The two degree-of-freedom chi-square test indicates that prog is a statistically significant predictor of daysabs.
\item The null deviance is calculated from an intercept-only model with 313 degrees of freedom. Then we see the residual deviance, the deviance from the full model. We are also shown the AIC and 2*log likelihood. 
\item The theta parameter shown is the dispersion parameter. Note that R parameterizes this differently from SAS, Stata, and SPSS. 
\item The R parameter (theta) is equal to the inverse of the dispersion parameter (alpha) estimated in these other software packages. Thus, the theta value of 1.033 seen here is equivalent to the 0.968 value seen in the Stata Negative Binomial Data Analysis Example because 1/0.968 = 1.033. 
\end{itemize}
\newpage
\subsection{Checking model assumption}

As we mentioned earlier, negative binomial models assume the conditional means are not equal to the conditional variances. This inequality is captured by estimating a dispersion parameter (not shown in the output) that is held constant in a Poisson model. Thus, the Poisson model is actually nested in the negative binomial model. We can then use a likelihood ratio test to compare these two and test this model assumption. To do this, we will run our model as a Poisson.
\begin{framed}
\begin{verbatim}
m3 <- glm(daysabs ~ math + prog, family = "poisson", data = dat)
pchisq(2 * (logLik(m1) - logLik(m3)), df = 1, lower.tail = FALSE)

## 'log Lik.' 2.157e-203 (df=5)
\end{verbatim}
\end{framed}
%================================================================================================= %



In this example the associated chi-squared value is 926.03 with one degree of freedom.
This strongly suggests the negative binomial model, estimating the dispersion parameter, is more appropriate than the Poisson model.
We can get the confidence intervals for the coefficients by profiling the likelihood function.



(est <- cbind(Estimate = coef(m1), confint(m1)))

## Waiting for profiling to be done...

##                 Estimate   2.5 %    97.5 %
## (Intercept)     2.615265  2.2421  3.012936
## math           -0.005993 -0.0109 -0.001067
## progAcademic   -0.440760 -0.8101 -0.092643
## progVocational -1.278651 -1.6835 -0.890078

We might be interested in looking at incident rate ratios rather than coefficients. To do this, we can exponentiate our model coefficients. The same applies to the confidence intervals.



exp(est)

##                Estimate  2.5 %  97.5 %
## (Intercept)     13.6708 9.4127 20.3470
## math             0.9940 0.9892  0.9989
## progAcademic     0.6435 0.4448  0.9115
## progVocational   0.2784 0.1857  0.4106

The output above indicates that the incident rate for prog = 2 is 0.64 times the incident rate for the reference group (prog = 1). Likewise, the incident rate for prog = 3 is 0.28 times the incident rate for the reference group holding the other variables constant. The percent change in the incident rate of daysabs is a 1% decrease for every unit increase in math.

The form of the model equation for negative binomial regression is the same as that for Poisson regression. The log of the outcome is predicted with a linear combination of the predictors:
\[ ln(\widehat{daysabs_i}) = Intercept + b_1(prog_i = 2) + b_2(prog_i = 3) + b_3math_i \] \[ \therefore \] \[ \widehat{daysabs_i} = e^{Intercept + b_1(prog_i = 2) + b_2(prog_i = 3) + b_3math_i} = e^{Intercept}e^{b_1(prog_i = 2)}e^{b_2(prog_i = 3)}e^{b_3math_i} \]
The coefficients have an additive effect in the \(ln(y)\) scale and the IRR have a multiplicative effect in the y scale. The dispersion parameter in negative binomial regression does not effect the expected counts, but it does effect the estimated variance of the expected counts. More details can be found in the Modern Applied Statistics with S by W.N. Venables and B.D. Ripley (the book companion of the MASS package).

For additional information on the various metrics in which the results can be presented, and the interpretation of such, please see Regression Models for Categorical Dependent Variables Using Stata, Second Edition by J. Scott Long and Jeremy Freese (2006).

\subsection{Predicted values}

For assistance in further understanding the model, we can look at predicted counts for various levels of our predictors. Below we create new datasets with values of math and prog and then use the predict command to calculate the predicted number of events.

First, we can look at predicted counts for each value of prog while holding math at its mean. To do this, we create a new dataset with the combinations of prog and math for which we would like to find predicted values, then use the predict command.



newdata1 <- data.frame(math = mean(dat$math), prog = factor(1:3, levels = 1:3, 
labels = levels(dat$prog)))
newdata1$phat <- predict(m1, newdata1, type = "response")
newdata1

##    math       prog   phat
## 1 48.27    General 10.237
## 2 48.27   Academic  6.588
## 3 48.27 Vocational  2.850

In the output above, we see that the predicted number of events (e.g., days absent) for a general program is about 10.24, holding math at its mean. The predicted number of events for an academic program is lower at 6.59, and the predicted number of events for a vocational program is about 2.85.

Below we will obtain the mean predicted number of events for values of math across its entire range for each level of prog and graph these.



newdata2 <- data.frame(
math = rep(seq(from = min(dat$math), to = max(dat$math), length.out = 100), 3),
prog = factor(rep(1:3, each = 100), levels = 1:3, labels =
levels(dat$prog)))

newdata2 <- cbind(newdata2, predict(m1, newdata2, type = "link", se.fit=TRUE))
newdata2 <- within(newdata2, {
	DaysAbsent <- exp(fit)
	LL <- exp(fit - 1.96 * se.fit)
	UL <- exp(fit + 1.96 * se.fit)
})

\begin{framed}
\begin{verbatim}
ggplot(newdata2, aes(math, DaysAbsent)) +
geom_ribbon(aes(ymin = LL, ymax = UL, fill = prog), alpha = .25) +
geom_line(aes(colour = prog), size = 2) +
labs(x = "Math Score", y = "Predicted Days Absent")

\end{verbatim}
\end{framed}

The graph shows the expected count across the range of math scores, for each type of program along with 95 percent confidence intervals. Note that the lines are not straight because this is a log linear model, and what is plotted are the expected values, not the log of the expected values.

\subsection{Things to consider}
•It is not recommended that negative binomial models be applied to small samples. 
•One common cause of over-dispersion is excess zeros by an additional data generating process. In this situation, zero-inflated model should be considered.
•If the data generating process does not allow for any 0s (such as the number of days spent in the hospital), then a zero-truncated model may be more appropriate.
•Count data often have an exposure variable, which indicates the number of times the event could have happened. This variable should be incorporated into your negative binomial regression model with the use of the offset option. See the \texttt{glm} documentation for details.
•The outcome variable in a negative binomial regression cannot have negative numbers.
•You will need to use the \texttt{m1\$resid} command to obtain the residuals from our model to check other assumptions of the negative binomial model (see Cameron and Trivedi (1998) and Dupont (2002) for more information).

%======================================================%
\end{document}
\section*{Stata}
Negative Binomial Regression
Version info: Code for this page was tested in Stata 12. 

Negative binomial regression is for modeling count variables, usually for over-dispersed count outcome variables.

Please note: The purpose of this page is to show how to use various data analysis commands.  It does not cover all aspects of the research process which researchers are expected to do.  In particular, it does not cover data cleaning and checking, verification of assumptions, model diagnostics or potential follow-up analyses. 

Examples of negative binomial regression
Example 1.  School administrators study the attendance behavior of high school juniors at two schools.  Predictors of the number of days of absence include the type of program in which the student is enrolled and a standardized test in math. 

Example 2.  A health-related researcher is studying the number of hospital visits in past 12 months by senior citizens in a community based on the characteristics of the individuals and the types of health plans under which each one is covered.  

Description of the data
Let's pursue Example 1 from above.

We have attendance data on 314 high school juniors from two urban high schools in the file nb_data.dta.  The response variable of interest is days absent, daysabs.  The variable math is the standardized math score for each student.  The variable prog is a three-level nominal variable indicating the type of instructional program in which the student is enrolled.

Let's look at the data.  It is always a good idea to start with descriptive statistics and plots.

use http://www.ats.ucla.edu/stat/stata/dae/nb_data, clear

summarize daysabs math
Variable |       Obs        Mean    Std. Dev.       Min        Max
-------------+--------------------------------------------------------
daysabs |       314    5.955414    7.036958          0         35
math |       314    48.26752    25.36239          1         99

histogram daysabs, discrete freq scheme(s1mono)



Each variable has 314 valid observations and their distributions seem quite reasonable. The unconditional mean of our outcome variable is much lower than its variance. 

Let's continue with our description of the variables in this dataset. The table below shows the average numbers of days absent by program type and seems to suggest that program type is a good candidate for predicting the number of days absent, our outcome variable, because the mean value of the outcome appears to vary by prog. The variances within each level of prog are higher than the means within each level. These are the conditional means and variances. These differences suggest that over-dispersion is present and that a Negative Binomial model would be appropriate. 

tabstat daysabs, by(prog) stats(mean v n)
Summary for variables: daysabs
by categories of: prog 

prog |      mean  variance         N
---------+------------------------------
1 |     10.65  67.25897        40
2 |  6.934132  55.44744       167
3 |  2.672897  13.93916       107
---------+------------------------------
Total |  5.955414  49.51877       314
----------------------------------------

Analysis methods you might consider
Below is a list of some analysis methods you may have encountered.  Some of the methods listed are quite reasonable, while others have either fallen out of favor or have limitations.  

Negative binomial regression - Negative binomial regression can be used for over-dispersed count data, that is when the conditional variance exceeds the conditional mean. It can be considered as a generalization of Poisson regression since it has the same mean structure as Poisson regression and it has an extra parameter to model the over-dispersion. If the conditional distribution of the outcome variable is over-dispersed, the confidence intervals for the Negative binomial regression are likely to be narrower as compared to those from a Poisson regression model.
Poisson regression - Poisson regression is often used for modeling count data. Poisson regression has a number of extensions useful for count models. 
Zero-inflated regression model - Zero-inflated models attempt to account for excess zeros.  In other words, two kinds of zeros are thought to exist in the data, "true zeros" and "excess zeros".  Zero-inflated models estimate two equations simultaneously, one for the count model and one for the excess zeros.
OLS regression - Count outcome variables are sometimes log-transformed and analyzed using OLS regression.  Many issues arise with this approach, including loss of data due to undefined values generated by taking the log of zero (which is undefined), as well as the lack of capacity to model the dispersion.
Negative binomial regression analysis
Below we use the nbreg command to estimate a negative binomial regression model.  The i. before prog indicates that it is a factor variable (i.e., categorical variable), and that it should be included in the model as a series of indicator variables.  

nbreg daysabs math i.prog

Fitting Poisson model:

Iteration 0:   log likelihood = -1328.6751 
Iteration 1:   log likelihood = -1328.6425 
Iteration 2:   log likelihood = -1328.6425 

Fitting constant-only model:

Iteration 0:   log likelihood = -899.27009 
Iteration 1:   log likelihood = -896.47264 
Iteration 2:   log likelihood = -896.47237 
Iteration 3:   log likelihood = -896.47237 

Fitting full model:

Iteration 0:   log likelihood = -870.49809 
Iteration 1:   log likelihood = -865.90381 
Iteration 2:   log likelihood = -865.62942 
Iteration 3:   log likelihood =  -865.6289 
Iteration 4:   log likelihood =  -865.6289 

Negative binomial regression                      Number of obs   =        314
LR chi2(3)      =      61.69
Dispersion     = mean                             Prob > chi2     =     0.0000
Log likelihood = -865.6289                        Pseudo R2       =     0.0344

------------------------------------------------------------------------------
daysabs |      Coef.   Std. Err.      z    P>|z|     [95% Conf. Interval]
-------------+----------------------------------------------------------------
math |   -.005993   .0025072    -2.39   0.017     -.010907    -.001079
|
prog |
2  |    -.44076    .182576    -2.41   0.016    -.7986025   -.0829175
3  |  -1.278651   .2019811    -6.33   0.000    -1.674526    -.882775
|
_cons |   2.615265   .1963519    13.32   0.000     2.230423    3.000108
-------------+----------------------------------------------------------------
/lnalpha |  -.0321895   .1027882                     -.2336506    .1692717
-------------+----------------------------------------------------------------
alpha |   .9683231   .0995322                      .7916384    1.184442
------------------------------------------------------------------------------
Likelihood-ratio test of alpha=0:  chibar2(01) =  926.03 Prob>=chibar2 = 0.000

The output begins the iteration log. We can see that it starts with fitting a Poisson model, then a null model (intercept only model) and finally the negative binomial model. Since it uses maximum likelihood estimate, it iterates until the change in the log likelihood is sufficiently small. The last value in the iteration log is the final value of the log likelihood for the full model and is displayed again.  The log likelihood can be used to compare models.
The header information is presented next.  On the right-hand side, the number of observations used in the analysis (314) is given, along with the Wald chi-square statistic with three degrees of freedom for the full model, followed by the p-value for the chi-square. This is a test that all of the estimated coefficients are equal to zero--a test of the model as a whole. From the p-value, we can see that the model is statistically significant. The header also includes a pseudo-R2, which is 0.03 in this example.
Below the header you will find the negative binomial regression coefficients for each of the variables, along with standard errors, z-scores, p-values and 95% confidence intervals for the coefficients.  The variable math has a coefficient of -0.006, which is statistically significant.  This means that for each one-unit increase on math, the expected log count of the number of days absent decreases by 0.006. The indicator variable 2.prog is the expected difference in log count between group 2 (prog=2) and the reference group (prog=1).  The expected log count for level 2 of prog is 0.44 lower than the expected log count for level 1. The indicator variable 3.prog is the expected difference in log count between group 3 (prog=3) and the reference group (prog=1).  The expected log count for level 3 of prog is 1.28 lower than the expected log count for level 1.  To determine if prog itself, overall, is statistically significant, we can use the test command to obtain the two degrees-of-freedom test of this variable. The two degree-of-freedom chi-square test indicates that prog is a statistically significant predictor of daysabs.
test 2.prog 3.prog

( 1)  [daysabs]2.prog = 0
( 2)  [daysabs]3.prog = 0

chi2(  2) =   49.21
Prob > chi2 =    0.0000

Additionally, the log-transformed over-dispersion parameter (/lnalpha) is estimated and is displayed along with the untransformed value. A Poisson model is one in which this alpha value is constrained to zero. Stata finds the maximum likelihood estimate of the log of alpha and then calculates alpha from this. This means that alpha is always greater than zero and that Stata's nbreg only allows for overdispersion (variance greater than the mean). 
Below the table of coefficients, you will find a likelihood ratio test that alpha equals zero--the likelihood ratio test comparing this model to a Poisson model.  In this example the associated chi-squared value is 926.03 with one degree of freedom.  This strongly suggests that alpha is non-zero and the negative binomial model is more appropriate than the Poisson model. 
We can also see the results as incident rate ratios by using the irr option.  

nbreg, irr
Negative binomial regression                      Number of obs   =        314
LR chi2(3)      =      61.69
Dispersion     = mean                             Prob > chi2     =     0.0000
Log likelihood = -865.6289                        Pseudo R2       =     0.0344

------------------------------------------------------------------------------
daysabs |        IRR   Std. Err.      z    P>|z|     [95% Conf. Interval]
-------------+----------------------------------------------------------------
math |   .9940249   .0024922    -2.39   0.017     .9891523    .9989216
|
prog |
2  |   .6435471   .1174963    -2.41   0.016     .4499573     .920427
3  |   .2784127   .0562341    -6.33   0.000     .1873969    .4136335
-------------+----------------------------------------------------------------
/lnalpha |  -.0321895   .1027882                     -.2336506    .1692717
-------------+----------------------------------------------------------------
alpha |   .9683231   .0995322                      .7916384    1.184442
------------------------------------------------------------------------------
Likelihood-ratio test of alpha=0:  chibar2(01) =  926.03 Prob>=chibar2 = 0.000

The output above indicates that the incident rate for 2.prog is 0.64 times the incident rate for the reference group (1.prog). Likewise, the incident rate for 3.prog is 0.28 times the incident rate for the reference group holding the other variables constant. The percent change in the incident rate of daysabs is a 1% decrease for every unit increase in math. 

The form of the model equation for negative binomial regression is the same as that for Poisson regression. The log of the outcome is predicted with a linear combination of the predictors:

log(daysabs) = Intercept + b1(prog=2) + b2(prog=3) + b3math.
This implies:

daysabs = exp(Intercept + b1(prog=2) + b2(prog=3)+ b3math) = exp(Intercept) * exp(b1(prog=2)) * exp(b2(prog=3)) * exp(b3math)
The coefficients have an additive effect in the log(y) scale and the IRR have a multiplicative effect in the y scale. The dispersion parameter alpha in negative binomial regression does not effect the expected counts, but it does effect the estimated variance of the expected counts. More details can be found in the Stata documentation.

For additional information on the various metrics in which the results can be presented, and the interpretation of such, please see Regression Models for Categorical Dependent Variables Using Stata, Second Edition by J. Scott Long and Jeremy Freese (2006).

To understand the model better, we can use the margins command. Below we use the margins command to calculate the predicted counts at each level of prog, holding all other variables (in this example, math) in the model at their means.

margins prog, atmeans

Adjusted predictions                              Number of obs   =        314
Model VCE    : OIM

Expression   : Predicted number of events, predict()
at           : math            =    48.26752 (mean)
1.prog          =    .1273885 (mean)
2.prog          =    .5318471 (mean)
3.prog          =    .3407643 (mean)

------------------------------------------------------------------------------
|            Delta-method
|     Margin   Std. Err.      z    P>|z|     [95% Conf. Interval]
-------------+----------------------------------------------------------------
prog |
1  |    10.2369   1.674445     6.11   0.000     6.955048    13.51875
2  |   6.587927   .5511718    11.95   0.000      5.50765    7.668204
3  |   2.850083   .3296496     8.65   0.000     2.203981    3.496184
------------------------------------------------------------------------------
In the output above, we see that the predicted number of events for level 1 of prog is about 10.24, holding math at its mean.  The predicted number of events for level 2 of prog is lower at 6.59, and the predicted number of events for level 3 of prog is about 2.85. Note that the predicted count of level 2 of prog is (6.587927/10.2369) = 0.64 times the predicted count for level 1 of prog. This matches what we saw in the IRR output table.

Below we will obtain the predicted number of events for values of math that range from 0 to 100 in increments of 20.

margins, at(math=(0(20)100)) vsquish

Predictive margins                                Number of obs   =        314
Model VCE    : OIM

Expression   : Predicted number of events, predict()
1._at        : math            =           0
2._at        : math            =          20
3._at        : math            =          40
4._at        : math            =          60
5._at        : math            =          80
6._at        : math            =         100

------------------------------------------------------------------------------
|            Delta-method
|     Margin   Std. Err.      z    P>|z|     [95% Conf. Interval]
-------------+----------------------------------------------------------------
_at |
1  |   7.717607   .9993707     7.72   0.000     5.758876    9.676337
2  |   6.845863   .6132453    11.16   0.000     5.643924    8.047802
3  |   6.072587   .3986397    15.23   0.000     5.291268    6.853907
4  |   5.386657   .4039343    13.34   0.000      4.59496    6.178354
5  |   4.778206   .5226812     9.14   0.000      3.75377    5.802643
6  |   4.238483   .6474951     6.55   0.000     2.969416     5.50755
------------------------------------------------------------------------------

The table above shows that with prog at its observed values and math held at 0 for all observations, the average predicted count (or average number of days absent) is about 7.72; when math = 100, the average predicted count is about 4.24. If we compare the predicted counts at any two levels of math, like math = 20 and math = 40, we can see that the ratio is (6.072587/6.845863) = 0.887. This matches the IRR of 0.994 for a 20 unit change: 0.994^20 = 0.887.

The user-written fitstat command (as well as Stata's estat commands) can be used to obtain additional model fit information that may be helpful if you want to compare models.  You can type findit fitstat to download this program (see How can I use the findit command to search for programs and get additional help? for more information about using findit).

fitstat
Measures of Fit for nbreg of daysabs

Log-Lik Intercept Only:       -896.472   Log-Lik Full Model:           -865.629
D(308):                       1731.258   LR(3):                          61.687
Prob > LR:                       0.000
McFadden's R2:                   0.034   McFadden's Adj R2:               0.028
ML (Cox-Snell) R2:               0.178   Cragg-Uhler(Nagelkerke) R2:      0.179
AIC:                             5.552   AIC*n:                        1743.258
BIC:                           -39.555   BIC':                          -44.439
BIC used by Stata:            1760.005   AIC used by Stata:            1741.258

You can graph the predicted number of events with the commands below.  The graph indicates that the most days absent are predicted for those in the academic program 1, especially if the studR Data Analysis Examples: Negative Binomial Regression

Negative binomial regression is for modeling count variables, usually for over-dispersed count outcome variables.

This page uses the following packages. Make sure that you can load them before trying to run the examples on this page. If you do not have a package installed, run: install.packages("packagename"), or if you see the version is out of date, run: update.packages().



require(foreign)
require(ggplot2)
require(MASS)

Version info: Code for this page was tested in R Under development (unstable) (2013-01-06 r61571)
On: 2013-01-22
With: MASS 7.3-22; ggplot2 0.9.3; foreign 0.8-52; knitr 1.0.5 

Please note: The purpose of this page is to show how to use various data analysis commands. It does not cover all aspects of the research process which researchers are expected to do. In particular, it does not cover data cleaning and checking, verification of assumptions, model diagnostics or potential follow-up analyses.

Examples of negative binomial regression

Example 1. School administrators study the attendance behavior of high school juniors at two schools. Predictors of the number of days of absence include the type of program in which the student is enrolled and a standardized test in math. 

Example 2. A health-related researcher is studying the number of hospital visits in past 12 months by senior citizens in a community based on the characteristics of the individuals and the types of health plans under which each one is covered.

Description of the data

Let's pursue Example 1 from above.

We have attendance data on 314 high school juniors from two urban high schools in the file nb_data. The response variable of interest is days absent, daysabs. The variable math gives the standardized math score for each student. The variable prog is a three-level nominal variable indicating the type of instructional program in which the student is enrolled.

Let's look at the data. It is always a good idea to start with descriptive statistics and plots.



dat <- read.dta("http://www.ats.ucla.edu/stat/stata/dae/nb_data.dta")
dat <- within(dat, {
	prog <- factor(prog, levels = 1:3, labels = c("General", "Academic", "Vocational"))
	id <- factor(id)
})

summary(dat)

##        id         gender         math         daysabs     
##  1001   :  1   female:160   Min.   : 1.0   Min.   : 0.00  
##  1002   :  1   male  :154   1st Qu.:28.0   1st Qu.: 1.00  
##  1003   :  1                Median :48.0   Median : 4.00  
##  1004   :  1                Mean   :48.3   Mean   : 5.96  
##  1005   :  1                3rd Qu.:70.0   3rd Qu.: 8.00  
##  1006   :  1                Max.   :99.0   Max.   :35.00  
##  (Other):308                                              
##          prog    
##  General   : 40  
##  Academic  :167  
##  Vocational:107  
##                  
##                  
##                  
## 

ggplot(dat, aes(daysabs, fill = prog)) + geom_histogram(binwidth = 1) + facet_grid(prog ~ 
., margins = TRUE, scales = "free")



Each variable has 314 valid observations and their distributions seem quite reasonable. The unconditional mean of our outcome variable is much lower than its variance. 

Let's continue with our description of the variables in this dataset. The table below shows the average numbers of days absent by program type and seems to suggest that program type is a good candidate for predicting the number of days absent, our outcome variable, because the mean value of the outcome appears to vary by prog. The variances within each level of prog are higher than the means within each level. These are the conditional means and variances. These differences suggest that over-dispersion is present and that a Negative Binomial model would be appropriate. 



with(dat, tapply(daysabs, prog, function(x) {
	sprintf("M (SD) = %1.2f (%1.2f)", mean(x), sd(x))
}))

##                 General                Academic              Vocational 
## "M (SD) = 10.65 (8.20)"  "M (SD) = 6.93 (7.45)"  "M (SD) = 2.67 (3.73)"

Analysis methods you might consider

Below is a list of some analysis methods you may have encountered. Some of the methods listed are quite reasonable, while others have either fallen out of favor or have limitations. 
•Negative binomial regression -Negative binomial regression can be used for over-dispersed count data, that is when the conditional variance exceeds the conditional mean. It can be considered as a generalization of Poisson regression since it has the same mean structure as Poisson regression and it has an extra parameter to model the over-dispersion. If the conditional distribution of the outcome variable is over-dispersed, the confidence intervals for the Negative binomial regression are likely to be narrower as compared to those from a Poisson regression model.
•Poisson regression - Poisson regression is often used for modeling count data. Poisson regression has a number of extensions useful for count models. 
•Zero-inflated regression model - Zero-inflated models attempt to account for excess zeros. In other words, two kinds of zeros are thought to exist in the data, "true zeros" and "excess zeros". Zero-inflated models estimate two equations simultaneously, one for the count model and one for the excess zeros.
•OLS regression - Count outcome variables are sometimes log-transformed and analyzed using OLS regression. Many issues arise with this approach, including loss of data due to undefined values generated by taking the log of zero (which is undefined), as well as the lack of capacity to model the dispersion.

Negative binomial regression analysis

Below we use the glm.nb function from the MASS package to estimate a negative binomial regression.



summary(m1 <- glm.nb(daysabs ~ math + prog, data = dat))

## 
## Call:
## glm.nb(formula = daysabs ~ math + prog, data = dat, init.theta = 1.032713156, 
##     link = log)
## 
## Deviance Residuals: 
##    Min      1Q  Median      3Q     Max  
## -2.155  -1.019  -0.369   0.229   2.527  
## 
## Coefficients:
##                Estimate Std. Error z value Pr(>|z|)    
## (Intercept)     2.61527    0.19746   13.24  < 2e-16 ***
## math           -0.00599    0.00251   -2.39    0.017 *  
## progAcademic   -0.44076    0.18261   -2.41    0.016 *  
## progVocational -1.27865    0.20072   -6.37  1.9e-10 ***
## ---
## Signif. codes:  0 '***' 0.001 '**' 0.01 '*' 0.05 '.' 0.1 ' ' 1
## 
## (Dispersion parameter for Negative Binomial(1.033) family taken to be 1)
## 
##     Null deviance: 427.54  on 313  degrees of freedom
## Residual deviance: 358.52  on 310  degrees of freedom
## AIC: 1741
## 
## Number of Fisher Scoring iterations: 1
## 
## 
##               Theta:  1.033 
##           Std. Err.:  0.106 
## 
##  2 x log-likelihood:  -1731.258
•R first displays the call and the deviance residuals. Next, we see the regression coefficients for each of the variables, along with standard errors, z-scores, and p-values. The variable math has a coefficient of -0.006, which is statistically significant. This means that for each one-unit increase in math, the expected log count of the number of days absent decreases by 0.006. The indicator variable shown as progAcademic is the expected difference in log count between group 2 and the reference group (prog=1). The expected log count for level 2 of prog is 0.44 lower than the expected log count for level 1. The indicator variable for progVocational is the expected difference in log count between group 3 and the reference group.The expected log count for level 3 of prog is 1.28 lower than the expected log count for level 1. To determine if prog itself, overall, is statistically significant, we can compare a model with and without prog. The reason it is important to fit separate models, is that unless we do, the overdispersion parameter is held constant.



m2 <- update(m1, . ~ . - prog)
anova(m1, m2)

## Likelihood ratio tests of Negative Binomial Models
## 
## Response: daysabs
##         Model  theta Resid. df    2 x log-lik.   Test    df LR stat.
## 1        math 0.8559       312           -1776                      
## 2 math + prog 1.0327       310           -1731 1 vs 2     2    45.05
##     Pr(Chi)
## 1          
## 2 1.652e-10
•The two degree-of-freedom chi-square test indicates that prog is a statistically significant predictor of daysabs.
•The null deviance is calculated from an intercept-only model with 313 degrees of freedom. Then we see the residual deviance, the deviance from the full model. We are also shown the AIC and 2*log likelihood. 
•The theta parameter shown is the dispersion parameter. Note that R parameterizes this differently from SAS, Stata, and SPSS. The R parameter (theta) is equal to the inverse of the dispersion parameter (alpha) estimated in these other software packages. Thus, the theta value of 1.033 seen here is equivalent to the 0.968 value seen in the Stata Negative Binomial Data Analysis Example because 1/0.968 = 1.033. 

Checking model assumption

As we mentioned earlier, negative binomial models assume the conditional means are not equal to the conditional variances. This inequality is captured by estimating a dispersion parameter (not shown in the output) that is held constant in a Poisson model. Thus, the Poisson model is actually nested in the negative binomial model. We can then use a likelihood ratio test to compare these two and test this model assumption. To do this, we will run our model as a Poisson.



m3 <- glm(daysabs ~ math + prog, family = "poisson", data = dat)
pchisq(2 * (logLik(m1) - logLik(m3)), df = 1, lower.tail = FALSE)

## 'log Lik.' 2.157e-203 (df=5)

In this example the associated chi-squared value is 926.03 with one degree of freedom. This strongly suggests the negative binomial model, estimating the dispersion parameter, is more appropriate than the Poisson model.

We can get the confidence intervals for the coefficients by profiling the likelihood function.



(est <- cbind(Estimate = coef(m1), confint(m1)))

## Waiting for profiling to be done...

##                 Estimate   2.5 %    97.5 %
## (Intercept)     2.615265  2.2421  3.012936
## math           -0.005993 -0.0109 -0.001067
## progAcademic   -0.440760 -0.8101 -0.092643
## progVocational -1.278651 -1.6835 -0.890078

We might be interested in looking at incident rate ratios rather than coefficients. To do this, we can exponentiate our model coefficients. The same applies to the confidence intervals.



exp(est)

##                Estimate  2.5 %  97.5 %
## (Intercept)     13.6708 9.4127 20.3470
## math             0.9940 0.9892  0.9989
## progAcademic     0.6435 0.4448  0.9115
## progVocational   0.2784 0.1857  0.4106

The output above indicates that the incident rate for prog = 2 is 0.64 times the incident rate for the reference group (prog = 1). Likewise, the incident rate for prog = 3 is 0.28 times the incident rate for the reference group holding the other variables constant. The percent change in the incident rate of daysabs is a 1% decrease for every unit increase in math.

The form of the model equation for negative binomial regression is the same as that for Poisson regression. The log of the outcome is predicted with a linear combination of the predictors:
\[ ln(\widehat{daysabs_i}) = Intercept + b_1(prog_i = 2) + b_2(prog_i = 3) + b_3math_i \] \[ \therefore \] \[ \widehat{daysabs_i} = e^{Intercept + b_1(prog_i = 2) + b_2(prog_i = 3) + b_3math_i} = e^{Intercept}e^{b_1(prog_i = 2)}e^{b_2(prog_i = 3)}e^{b_3math_i} \]
The coefficients have an additive effect in the \(ln(y)\) scale and the IRR have a multiplicative effect in the y scale. The dispersion parameter in negative binomial regression does not effect the expected counts, but it does effect the estimated variance of the expected counts. More details can be found in the Modern Applied Statistics with S by W.N. Venables and B.D. Ripley (the book companion of the MASS package).

For additional information on the various metrics in which the results can be presented, and the interpretation of such, please see Regression Models for Categorical Dependent Variables Using Stata, Second Edition by J. Scott Long and Jeremy Freese (2006).

Predicted values

For assistance in further understanding the model, we can look at predicted counts for various levels of our predictors. Below we create new datasets with values of math and prog and then use the predict command to calculate the predicted number of events.

First, we can look at predicted counts for each value of prog while holding math at its mean. To do this, we create a new dataset with the combinations of prog and math for which we would like to find predicted values, then use the predict command.



newdata1 <- data.frame(math = mean(dat$math), prog = factor(1:3, levels = 1:3, 
labels = levels(dat$prog)))
newdata1$phat <- predict(m1, newdata1, type = "response")
newdata1

##    math       prog   phat
## 1 48.27    General 10.237
## 2 48.27   Academic  6.588
## 3 48.27 Vocational  2.850

In the output above, we see that the predicted number of events (e.g., days absent) for a general program is about 10.24, holding math at its mean. The predicted number of events for an academic program is lower at 6.59, and the predicted number of events for a vocational program is about 2.85.

Below we will obtain the mean predicted number of events for values of math across its entire range for each level of prog and graph these.



newdata2 <- data.frame(
math = rep(seq(from = min(dat$math), to = max(dat$math), length.out = 100), 3),
prog = factor(rep(1:3, each = 100), levels = 1:3, labels =
levels(dat$prog)))

newdata2 <- cbind(newdata2, predict(m1, newdata2, type = "link", se.fit=TRUE))
newdata2 <- within(newdata2, {
	DaysAbsent <- exp(fit)
	LL <- exp(fit - 1.96 * se.fit)
	UL <- exp(fit + 1.96 * se.fit)
})

ggplot(newdata2, aes(math, DaysAbsent)) +
geom_ribbon(aes(ymin = LL, ymax = UL, fill = prog), alpha = .25) +
geom_line(aes(colour = prog), size = 2) +
labs(x = "Math Score", y = "Predicted Days Absent")



The graph shows the expected count across the range of math scores, for each type of program along with 95 percent confidence intervals. Note that the lines are not straight because this is a log linear model, and what is plotted are the expected values, not the log of the expected values.

Things to consider
•It is not recommended that negative binomial models be applied to small samples. 
•One common cause of over-dispersion is excess zeros by an additional data generating process. In this situation, zero-inflated model should be considered.
•If the data generating process does not allow for any 0s (such as the number of days spent in the hospital), then a zero-truncated model may be more appropriate.
•Count data often have an exposure variable, which indicates the number of times the event could have happened. This variable should be incorporated into your negative binomial regression model with the use of the offset option. See the glm documentation for details.
•The outcome variable in a negative binomial regression cannot have negative numbers.
•You will need to use the m1$resid command to obtain the residuals from our model to check other assumptions of the negative binomial model (see Cameron and Trivedi (1998) and Dupont (2002) for more information).

%======================================================%

Negative Binomial Regression
Version info: Code for this page was tested in Stata 12. 

Negative binomial regression is for modeling count variables, usually for over-dispersed count outcome variables.

Please note: The purpose of this page is to show how to use various data analysis commands.  It does not cover all aspects of the research process which researchers are expected to do.  In particular, it does not cover data cleaning and checking, verification of assumptions, model diagnostics or potential follow-up analyses. 

Examples of negative binomial regression
Example 1.  School administrators study the attendance behavior of high school juniors at two schools.  Predictors of the number of days of absence include the type of program in which the student is enrolled and a standardized test in math. 

Example 2.  A health-related researcher is studying the number of hospital visits in past 12 months by senior citizens in a community based on the characteristics of the individuals and the types of health plans under which each one is covered.  

Description of the data
Let's pursue Example 1 from above.

We have attendance data on 314 high school juniors from two urban high schools in the file nb_data.dta.  The response variable of interest is days absent, daysabs.  The variable math is the standardized math score for each student.  The variable prog is a three-level nominal variable indicating the type of instructional program in which the student is enrolled.

Let's look at the data.  It is always a good idea to start with descriptive statistics and plots.

use http://www.ats.ucla.edu/stat/stata/dae/nb_data, clear

summarize daysabs math
Variable |       Obs        Mean    Std. Dev.       Min        Max
-------------+--------------------------------------------------------
daysabs |       314    5.955414    7.036958          0         35
math |       314    48.26752    25.36239          1         99

histogram daysabs, discrete freq scheme(s1mono)



Each variable has 314 valid observations and their distributions seem quite reasonable. The unconditional mean of our outcome variable is much lower than its variance. 

Let's continue with our description of the variables in this dataset. The table below shows the average numbers of days absent by program type and seems to suggest that program type is a good candidate for predicting the number of days absent, our outcome variable, because the mean value of the outcome appears to vary by prog. The variances within each level of prog are higher than the means within each level. These are the conditional means and variances. These differences suggest that over-dispersion is present and that a Negative Binomial model would be appropriate. 

tabstat daysabs, by(prog) stats(mean v n)
Summary for variables: daysabs
by categories of: prog 

prog |      mean  variance         N
---------+------------------------------
1 |     10.65  67.25897        40
2 |  6.934132  55.44744       167
3 |  2.672897  13.93916       107
---------+------------------------------
Total |  5.955414  49.51877       314
----------------------------------------

Analysis methods you might consider
Below is a list of some analysis methods you may have encountered.  Some of the methods listed are quite reasonable, while others have either fallen out of favor or have limitations.  

Negative binomial regression - Negative binomial regression can be used for over-dispersed count data, that is when the conditional variance exceeds the conditional mean. It can be considered as a generalization of Poisson regression since it has the same mean structure as Poisson regression and it has an extra parameter to model the over-dispersion. If the conditional distribution of the outcome variable is over-dispersed, the confidence intervals for the Negative binomial regression are likely to be narrower as compared to those from a Poisson regression model.
Poisson regression - Poisson regression is often used for modeling count data. Poisson regression has a number of extensions useful for count models. 
Zero-inflated regression model - Zero-inflated models attempt to account for excess zeros.  In other words, two kinds of zeros are thought to exist in the data, "true zeros" and "excess zeros".  Zero-inflated models estimate two equations simultaneously, one for the count model and one for the excess zeros.
OLS regression - Count outcome variables are sometimes log-transformed and analyzed using OLS regression.  Many issues arise with this approach, including loss of data due to undefined values generated by taking the log of zero (which is undefined), as well as the lack of capacity to model the dispersion.
Negative binomial regression analysis
Below we use the nbreg command to estimate a negative binomial regression model.  The i. before prog indicates that it is a factor variable (i.e., categorical variable), and that it should be included in the model as a series of indicator variables.  

nbreg daysabs math i.prog

Fitting Poisson model:

Iteration 0:   log likelihood = -1328.6751 
Iteration 1:   log likelihood = -1328.6425 
Iteration 2:   log likelihood = -1328.6425 

Fitting constant-only model:

Iteration 0:   log likelihood = -899.27009 
Iteration 1:   log likelihood = -896.47264 
Iteration 2:   log likelihood = -896.47237 
Iteration 3:   log likelihood = -896.47237 

Fitting full model:

Iteration 0:   log likelihood = -870.49809 
Iteration 1:   log likelihood = -865.90381 
Iteration 2:   log likelihood = -865.62942 
Iteration 3:   log likelihood =  -865.6289 
Iteration 4:   log likelihood =  -865.6289 

Negative binomial regression                      Number of obs   =        314
LR chi2(3)      =      61.69
Dispersion     = mean                             Prob > chi2     =     0.0000
Log likelihood = -865.6289                        Pseudo R2       =     0.0344

------------------------------------------------------------------------------
daysabs |      Coef.   Std. Err.      z    P>|z|     [95% Conf. Interval]
-------------+----------------------------------------------------------------
math |   -.005993   .0025072    -2.39   0.017     -.010907    -.001079
|
prog |
2  |    -.44076    .182576    -2.41   0.016    -.7986025   -.0829175
3  |  -1.278651   .2019811    -6.33   0.000    -1.674526    -.882775
|
_cons |   2.615265   .1963519    13.32   0.000     2.230423    3.000108
-------------+----------------------------------------------------------------
/lnalpha |  -.0321895   .1027882                     -.2336506    .1692717
-------------+----------------------------------------------------------------
alpha |   .9683231   .0995322                      .7916384    1.184442
------------------------------------------------------------------------------
Likelihood-ratio test of alpha=0:  chibar2(01) =  926.03 Prob>=chibar2 = 0.000

The output begins the iteration log. We can see that it starts with fitting a Poisson model, then a null model (intercept only model) and finally the negative binomial model. Since it uses maximum likelihood estimate, it iterates until the change in the log likelihood is sufficiently small. The last value in the iteration log is the final value of the log likelihood for the full model and is displayed again.  The log likelihood can be used to compare models.
The header information is presented next.  On the right-hand side, the number of observations used in the analysis (314) is given, along with the Wald chi-square statistic with three degrees of freedom for the full model, followed by the p-value for the chi-square. This is a test that all of the estimated coefficients are equal to zero--a test of the model as a whole. From the p-value, we can see that the model is statistically significant. The header also includes a pseudo-R2, which is 0.03 in this example.
Below the header you will find the negative binomial regression coefficients for each of the variables, along with standard errors, z-scores, p-values and 95% confidence intervals for the coefficients.  The variable math has a coefficient of -0.006, which is statistically significant.  This means that for each one-unit increase on math, the expected log count of the number of days absent decreases by 0.006. The indicator variable 2.prog is the expected difference in log count between group 2 (prog=2) and the reference group (prog=1).  The expected log count for level 2 of prog is 0.44 lower than the expected log count for level 1. The indicator variable 3.prog is the expected difference in log count between group 3 (prog=3) and the reference group (prog=1).  The expected log count for level 3 of prog is 1.28 lower than the expected log count for level 1.  To determine if prog itself, overall, is statistically significant, we can use the test command to obtain the two degrees-of-freedom test of this variable. The two degree-of-freedom chi-square test indicates that prog is a statistically significant predictor of daysabs.
test 2.prog 3.prog

( 1)  [daysabs]2.prog = 0
( 2)  [daysabs]3.prog = 0

chi2(  2) =   49.21
Prob > chi2 =    0.0000

Additionally, the log-transformed over-dispersion parameter (/lnalpha) is estimated and is displayed along with the untransformed value. A Poisson model is one in which this alpha value is constrained to zero. Stata finds the maximum likelihood estimate of the log of alpha and then calculates alpha from this. This means that alpha is always greater than zero and that Stata's nbreg only allows for overdispersion (variance greater than the mean). 
Below the table of coefficients, you will find a likelihood ratio test that alpha equals zero--the likelihood ratio test comparing this model to a Poisson model.  In this example the associated chi-squared value is 926.03 with one degree of freedom.  This strongly suggests that alpha is non-zero and the negative binomial model is more appropriate than the Poisson model. 
We can also see the results as incident rate ratios by using the irr option.  

nbreg, irr
Negative binomial regression                      Number of obs   =        314
LR chi2(3)      =      61.69
Dispersion     = mean                             Prob > chi2     =     0.0000
Log likelihood = -865.6289                        Pseudo R2       =     0.0344

------------------------------------------------------------------------------
daysabs |        IRR   Std. Err.      z    P>|z|     [95% Conf. Interval]
-------------+----------------------------------------------------------------
math |   .9940249   .0024922    -2.39   0.017     .9891523    .9989216
|
prog |
2  |   .6435471   .1174963    -2.41   0.016     .4499573     .920427
3  |   .2784127   .0562341    -6.33   0.000     .1873969    .4136335
-------------+----------------------------------------------------------------
/lnalpha |  -.0321895   .1027882                     -.2336506    .1692717
-------------+----------------------------------------------------------------
alpha |   .9683231   .0995322                      .7916384    1.184442
------------------------------------------------------------------------------
Likelihood-ratio test of alpha=0:  chibar2(01) =  926.03 Prob>=chibar2 = 0.000

The output above indicates that the incident rate for 2.prog is 0.64 times the incident rate for the reference group (1.prog). Likewise, the incident rate for 3.prog is 0.28 times the incident rate for the reference group holding the other variables constant. The percent change in the incident rate of daysabs is a 1% decrease for every unit increase in math. 

The form of the model equation for negative binomial regression is the same as that for Poisson regression. The log of the outcome is predicted with a linear combination of the predictors:

log(daysabs) = Intercept + b1(prog=2) + b2(prog=3) + b3math.
This implies:

daysabs = exp(Intercept + b1(prog=2) + b2(prog=3)+ b3math) = exp(Intercept) * exp(b1(prog=2)) * exp(b2(prog=3)) * exp(b3math)
The coefficients have an additive effect in the log(y) scale and the IRR have a multiplicative effect in the y scale. The dispersion parameter alpha in negative binomial regression does not effect the expected counts, but it does effect the estimated variance of the expected counts. More details can be found in the Stata documentation.

For additional information on the various metrics in which the results can be presented, and the interpretation of such, please see Regression Models for Categorical Dependent Variables Using Stata, Second Edition by J. Scott Long and Jeremy Freese (2006).

To understand the model better, we can use the margins command. Below we use the margins command to calculate the predicted counts at each level of prog, holding all other variables (in this example, math) in the model at their means.

margins prog, atmeans

Adjusted predictions                              Number of obs   =        314
Model VCE    : OIM

Expression   : Predicted number of events, predict()
at           : math            =    48.26752 (mean)
1.prog          =    .1273885 (mean)
2.prog          =    .5318471 (mean)
3.prog          =    .3407643 (mean)

------------------------------------------------------------------------------
|            Delta-method
|     Margin   Std. Err.      z    P>|z|     [95% Conf. Interval]
-------------+----------------------------------------------------------------
prog |
1  |    10.2369   1.674445     6.11   0.000     6.955048    13.51875
2  |   6.587927   .5511718    11.95   0.000      5.50765    7.668204
3  |   2.850083   .3296496     8.65   0.000     2.203981    3.496184
------------------------------------------------------------------------------
In the output above, we see that the predicted number of events for level 1 of prog is about 10.24, holding math at its mean.  The predicted number of events for level 2 of prog is lower at 6.59, and the predicted number of events for level 3 of prog is about 2.85. Note that the predicted count of level 2 of prog is (6.587927/10.2369) = 0.64 times the predicted count for level 1 of prog. This matches what we saw in the IRR output table.

Below we will obtain the predicted number of events for values of math that range from 0 to 100 in increments of 20.

margins, at(math=(0(20)100)) vsquish

Predictive margins                                Number of obs   =        314
Model VCE    : OIM

Expression   : Predicted number of events, predict()
1._at        : math            =           0
2._at        : math            =          20
3._at        : math            =          40
4._at        : math            =          60
5._at        : math            =          80
6._at        : math            =         100

------------------------------------------------------------------------------
|            Delta-method
|     Margin   Std. Err.      z    P>|z|     [95% Conf. Interval]
-------------+----------------------------------------------------------------
_at |
1  |   7.717607   .9993707     7.72   0.000     5.758876    9.676337
2  |   6.845863   .6132453    11.16   0.000     5.643924    8.047802
3  |   6.072587   .3986397    15.23   0.000     5.291268    6.853907
4  |   5.386657   .4039343    13.34   0.000      4.59496    6.178354
5  |   4.778206   .5226812     9.14   0.000      3.75377    5.802643
6  |   4.238483   .6474951     6.55   0.000     2.969416     5.50755
------------------------------------------------------------------------------

The table above shows that with prog at its observed values and math held at 0 for all observations, the average predicted count (or average number of days absent) is about 7.72; when math = 100, the average predicted count is about 4.24. If we compare the predicted counts at any two levels of math, like math = 20 and math = 40, we can see that the ratio is (6.072587/6.845863) = 0.887. This matches the IRR of 0.994 for a 20 unit change: 0.994^20 = 0.887.

The user-written fitstat command (as well as Stata's estat commands) can be used to obtain additional model fit information that may be helpful if you want to compare models.  You can type findit fitstat to download this program (see How can I use the findit command to search for programs and get additional help? for more information about using findit).

fitstat
Measures of Fit for nbreg of daysabs

Log-Lik Intercept Only:       -896.472   Log-Lik Full Model:           -865.629
D(308):                       1731.258   LR(3):                          61.687
Prob > LR:                       0.000
McFadden's R2:                   0.034   McFadden's Adj R2:               0.028
ML (Cox-Snell) R2:               0.178   Cragg-Uhler(Nagelkerke) R2:      0.179
AIC:                             5.552   AIC*n:                        1743.258
BIC:                           -39.555   BIC':                          -44.439
BIC used by Stata:            1760.005   AIC used by Stata:            1741.258

You can graph the predicted number of events with the commands below.  The graph indicates that the most days absent are predicted for those in the academic program 1, especially if the student has a low math score.  The lowest number of predicted days absent is for those students in program 3.

predict c
sort math
twoway (line c math if prog==1) ///
(line c math if prog==2) ///
(line c math if prog==3), ///
ytitle("Predicted Mean Value of Response") ylabel(0(3)15 ,nogrid) ///
xtitle("Math Score") legend(order(1 "prog=1" 2 "prog=2" 3 "prog=3")) scheme(s1mono)



Things to consider
It is not recommended that negative binomial models be applied to small samples. 
One common cause of over-dispersion is excess zeros by an additional data generating process.  In this situation, zero-inflated model should be considered.
If the data generating process does not allow for any 0s (such as the number of days spent in the hospital), then a zero-truncated model may be more appropriate.
Count data often have an exposure variable, which indicates the number of times the event could have happened.  This variable should be incorporated into your negative binomial regression model with the use of the exp() option.
The outcome variable in a negative binomial regression cannot have negative numbers, and the exposure cannot have 0s.
You can also run a negative binomial model using the glm command with the log link and the binomial family.
You will need to use the glm command to obtain the residuals to check other assumptions of the negative binomial model (see Cameron and Trivedi (1998) and Dupont (2002) for more information).
Pseudo-R-squared:  Many different measures of pseudo-R-squared exist. They all attempt to provide information similar to that provided by R-squared in OLS regression; however, none of them can be interpreted exactly as R-squared in OLS regression is interpreted. For a discussion of various pseudo-R-squares, see Long and Freese (2006) or our FAQ page What are pseudo R-squareds? .
ent has a low math score.  The lowest number of predicted days absent is for those students in program 3.

predict c
sort math
twoway (line c math if prog==1) ///
(line c math if prog==2) ///
(line c math if prog==3), ///
ytitle("Predicted Mean Value of Response") ylabel(0(3)15 ,nogrid) ///
xtitle("Math Score") legend(order(1 "prog=1" 2 "prog=2" 3 "prog=3")) scheme(s1mono)



Things to consider
It is not recommended that negative binomial models be applied to small samples. 
One common cause of over-dispersion is excess zeros by an additional data generating process.  In this situation, zero-inflated model should be considered.
If the data generating process does not allow for any 0s (such as the number of days spent in the hospital), then a zero-truncated model may be more appropriate.
Count data often have an exposure variable, which indicates the number of times the event could have happened.  This variable should be incorporated into your negative binomial regression model with the use of the exp() option.
The outcome variable in a negative binomial regression cannot have negative numbers, and the exposure cannot have 0s.
You can also run a negative binomial model using the glm command with the log link and the binomial family.
You will need to use the glm command to obtain the residuals to check other assumptions of the negative binomial model (see Cameron and Trivedi (1998) and Dupont (2002) for more information).
Pseudo-R-squared:  Many different measures of pseudo-R-squared exist. They all attempt to provide information similar to that provided by R-squared in OLS regression; however, none of them can be interpreted exactly as R-squared in OLS regression is interpreted. For a discussion of various pseudo-R-squares, see Long and Freese (2006) or our FAQ page What are pseudo R-squareds? .
