\documentclass[caret-main.tex]{subfiles}
\begin{document}

\section{Predictive Modeling}

% - http://jmlr.org/papers/volume11/visweswaran10a/visweswaran10a.pdf

\begin{itemize}
\item Prediction is a central problem in machine learning that involves inducing a model from a set of
training instances that is then applied to future instances to predict a target variable of interest.
\item Several commonly used predictive algorithms, such as logistic regression, neural networks, decision
trees, and Bayesian networks, typically induce a single model from a training set of instances, with
the intent of applying it to all future instances. 


\item We call such a model a population-wide model because it is intended to be applied to an entire population of future instances. A population-wide
model is optimized to predict well on average when applied to expected future instances.
Recent research in machine learning has shown that inducing models that are specific to the
particular features of a given instance can improve predictive performances (Gottrup et al., 2005).
\item We call such a model an instance-specific model since it is constructed specifically for a particular
instance (case). 

\item The structure and parameters of an instance-specific model are specialized to the
particular features of an instance, so that it is optimized to predict especially well for that instance.
\item The goal of inducing an instance-specific model is to obtain optimal prediction for the instance at
hand. This is in contrast to the induction of a population-wide model where the goal is to obtain
optimal predictive performance on average on all future instances.
\end{itemize}


%----------------------------------------------------------------------------5
\end{document}
