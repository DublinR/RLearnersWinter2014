
%======================================================================= %
% Introduction to EVT
\section{Extreme Value Theory}
Extreme Value Theory is mathematical study of extreme values.
Extreme value theory is a branch of statistics dealing with the extreme deviations from the median of probability distributions. The general theory sets out to assess the type of probability distributions generated by processes. Extreme value theory is important for assessing risk for highly unusual events, such as 100-year floods.

The field of extreme value theory was pioneered by Leonard Tippett ($1902-1985$). Tippett was employed by the British Cotton Industry Research Association, where he worked to make cotton thread stronger. In his studies, he realized that the strength of a thread was controlled by the strength of its weakest fibers.

With the help of R. A. Fisher, Tippet obtained three asymptotic limits describing the distributions of extremes. The German mathematician Emil Julius Gumbel codified this theory in his 1958 book \textbf{\emph{Statistics of Extremes}}, including the Gumbel distributions that bear his name.
%----------------------------------------------------------------------------------------%
\section{Frequency Analysis}
%---http://ihpnagoyaforum.org/textbooks/TakaraLecture.pdf
%----------------------------------------------------------------------------------------%
\begin{itemize}
	\item Frequency Analysis ($FA$): Probabilistic description of hydrological extremes
	\item Extraction of extreme variables from data
	\item Choice of a distribution
	\item Parameter estimation
	\item Quantiles + uncertainty
\end{itemize}



\end{document}