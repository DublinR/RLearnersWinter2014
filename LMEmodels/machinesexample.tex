\documentclass[main.tex]{subfiles}
\begin{document}

% Pinheiro Bates 


\section{``Machines" Data Set -  Example}
%\section{Computation on a mixed effects model}

\citet{pb} describes an experiment whereby the productivity of six
randomly chosen workers are assessed three times on each of three
machines, yielding the 54 observations tabulated below.

% latex table generated in R 2.9.2 by xtable 1.5-5 package
% Wed Sep 16 13:56:04 2009
\begin{table}[h!]
\begin{center}
\begin{tabular}{|c|c|c|c||c|c|c|c|}
  \hline
Observation & Worker & Machine & score & Observation & Worker & Machine & score \\
  \hline
1 & 1 & A & 52.00 &	28 & 4 & B & 63.20 \\
  2 & 1 & A & 52.80 &	  29 & 4 & B & 62.80 \\
  3 & 1 & A & 53.10 &	  30 & 4 & B & 62.20 \\
  4 & 2 & A & 51.80 &	  31 & 5 & B & 64.80 \\
  5 & 2 & A & 52.80 &	  32 & 5 & B & 65.00 \\
  6 & 2 & A & 53.10 &	  33 & 5 & B & 65.40 \\
  7 & 3 & A & 60.00 &	  34 & 6 & B & 43.70 \\
  8 & 3 & A & 60.20 &	  35 & 6 & B & 44.20 \\
  9 & 3 & A & 58.40 &	  36 & 6 & B & 43.00 \\
  10 & 4 & A & 51.10 &	  37 & 1 & C & 67.50 \\
  11 & 4 & A & 52.30 &	  38 & 1 & C & 67.20 \\
  12 & 4 & A & 50.30 &	  39 & 1 & C & 66.90 \\
  13 & 5 & A & 50.90 &	  40 & 2 & C & 61.50 \\
  14 & 5 & A & 51.80 &	  41 & 2 & C & 61.70 \\
  15 & 5 & A & 51.40 &	  42 & 2 & C & 62.30 \\
  16 & 6 & A & 46.40 &	  43 & 3 & C & 70.80 \\
  17 & 6 & A & 44.80 &	  44 & 3 & C & 70.60 \\
  18 & 6 & A & 49.20 &	  45 & 3 & C & 71.00 \\
  19 & 1 & B & 62.10 &	  46 & 4 & C & 64.10 \\
  20 & 1 & B & 62.60 &	  47 & 4 & C & 66.20 \\
  21 & 1 & B & 64.00 &	  48 & 4 & C & 64.00 \\
  22 & 2 & B & 59.70 &	  49 & 5 & C & 72.10 \\
  23 & 2 & B & 60.00 &	  50 & 5 & C & 72.00 \\
  24 & 2 & B & 59.00 &	  51 & 5 & C & 71.10 \\
  25 & 3 & B & 68.60 &	  52 & 6 & C & 62.00 \\
  26 & 3 & B & 65.80 &	  53 & 6 & C & 61.40 \\
  27 & 3 & B & 69.70 &	  54 & 6 & C & 60.50 \\

   \hline
\end{tabular}
\caption{Machines Data , Pinheiro Bates}
\end{center}
\end{table}
(Overall mean score = $59.65$, mean on machine A = $52.35$ , mean
on machine B = $60.32$, mean on machine C = $66.27$)
\newpage

The `worker' factor is modelled with random effects($u_{i}$),
whereas the `machine' factor is modelled with fixed effects
($\beta_{j}$). Due to the repeated nature of the data, interaction
effects between these factors are assumed to be extant, and shall
be examined accordingly. The interaction effect in this case
($\tau_{ij}$) describes whether the effect of changing from one
machine to another is different for each worker. The productivity
score $y_{ijk}$ is the $k$th observation taken on worker $i$ on
machine $j$, and is formulated
 as follows;

\begin{equation}
y_{ijk} = \beta_{j} + u_{i} + \tau_{ij} + \epsilon_{ijk}
\end{equation}
\begin{equation*}
u_{i} \sim N(0, \sigma^{2}_{u}), \epsilon_{ijk} \sim N(0,
\sigma^{2}), \tau_{i} \sim N(0, \sigma^{2}_{\tau})
\end{equation*}

The `nlme' package is incorporated into the R programming to
perform linear mixed model calculations. For the `Machines' data,
\citet{pb} use the following code, with the hierarchical structure
specified in the last argument.
\begin{verbatim}
lme(score~Machine, data=Machines, random=~1|Worker/Machine)
\end{verbatim}


The output of the R computation is given below.
\begin{verbatim}
Linear mixed-effects model fit by REML
  Data: Machines
  Log-restricted-likelihood: -107.8438
  Fixed: score ~ Machine
(Intercept)    MachineB    MachineC
  52.355556    7.966667   13.916667

Random effects:
 Formula: ~1 | Worker
        (Intercept)
StdDev:     4.78105

 Formula: ~1 | Machine %in% Worker
        (Intercept)  Residual
StdDev:    3.729532 0.9615771

Number of Observations: 54 Number of Groups:
             Worker Machine %in% Worker
                  6                  18

\end{verbatim}

\newpage
The crucial pieces of information given in the programme output
are the estimates of the intercepts for each of the three
machines. Machine A, which is treated as a control case, is
estimated to have an intercept of 52.35. The intercept estimates
for machines B and C are found to be $60.32$ and $66.27$ (by
adding the values 7.96 and 13.91 to 52.35 respectively). Estimate
for the variance components are also given; $\sigma^{2}_{u} =
(4.78)^{2}$ , $\sigma^{2}_{\tau} = (3.73)^{2}$ and
$\sigma^{2}_{\epsilon} = (0.96)^{2}$.

\end{document}