


\subsection{Priors}In the case of a categorical response (classification problem), minimizing costs amounts to minimizing the proportion of misclassified cases when priors are taken to be proportional to the class sizes and when misclassification costs are taken to be equal for every class.

The a priori probabilities used in minimizing costs can greatly affect the classification of cases or objects. Therefore, care has to be taken while using the priors. If differential base rates are not of interest for the study, or if we know that there are about an equal number of cases in each class, then we would use equal priors. If the differential base rates are reflected in the class sizes (as they would be, if the sample is a probability sample), then we would use priors estimated by the class proportions of the sample.
 Finally, if we have specific knowledge about the base rates (for example, based on previous research), then we would specify priors in accordance with that knowledge The general point is that the relative size of the priors assigned to each class can be used to "adjust" the importance of misclassifications for each class. However, no priors are required when we are building a regression tree.
%-------------------------------------------------------------------------------%
\subsubsection{Misclassification costs}
Sometimes more accurate classification of the response is desired for some classes than others for reasons not related to the relative class sizes. If the criterion for predictive accuracy is Misclassification costs, then minimizing costs would amount to minimizing the proportion of misclassified cases when priors are considered proportional to the class sizes and misclassification costs are taken to be equal for every class.

\subsubsection{Case weights} Case weights are treated strictly as case multipliers. For example, the misclassification rates from an analysis of an aggregated data set using case weights will be identical to the misclassification rates from the same analysis where the cases are replicated the specified number of times in the data file.

However, note that the use of case weights for aggregated data sets in classification problems is related to the issue of minimizing costs. Interestingly, as an alternative to using case weights for aggregated data sets, we could specify appropriate priors and/or misclassification costs and produce the same results while avoiding the additional processing required to analyze multiple cases with the same values for all variables. Suppose that in an aggregated data set with two classes having an equal number of cases, there are case weights of 2 for all cases in the first class, and case weights of 3 for all cases in the second class. If we specified priors of .4 and .6, respectively, specified equal misclassification costs, and analyzed the data without case weights, we will get the same misclassification rates as we would get if we specified priors estimated by the class sizes, specified equal misclassification costs, and analyzed the aggregated data set using the case weights. We would also get the same misclassification rates if we specified priors to be equal, specified the costs of misclassifying class 1 cases as class 2 cases to be 2/3 of the costs of misclassifying class 2 cases as class 1 cases, and analyzed the data without case weights.

%-------------------------------------------------------------%
\subsection{Selecting Splits}

The second basic step in classification and regression trees is to select the splits on the predictor variables that are used to predict membership in classes of the categorical dependent variables, or to predict values of the continuous dependent (response) variable. In general terms, the split at each node will be found that will generate the greatest improvement in predictive accuracy. 

This is usually measured with some type of node impurity measure, which provides an indication of the relative homogeneity (the inverse of impurity) of cases in the terminal nodes. If all cases in each terminal node show identical values, then node impurity is minimal, homogeneity is maximal, and prediction is perfect (at least for the cases used in the computations; predictive validity for new cases is of course a different matter...).

For classification problems, CART gives the user the choice of several impurity measures: The Gini index, Chi-square, or G-square. The Gini index of node impurity is the measure most commonly chosen for classification-type problems. As an impurity measure, it reaches a value of zero when only one class is present at a node. 

With priors estimated from class sizes and equal misclassification costs, the Gini measure is computed as the sum of products of all pairs of class proportions for classes present at the node; it reaches its maximum value when class sizes at the node are equal; the Gini index is equal to zero if all cases in a node belong to the same class. The Chi-square measure is similar to the standard Chi-square value computed for the expected and observed classifications (with priors adjusted for misclassification cost), and the G-square measure is similar to the maximum-likelihood 

Chi-square (as for example computed in the Log-Linear module).

For regression-type problems, a least-squares deviation criterion (similar to what is computed in least squares regression) is automatically used. 
%-------------------------------------------------------------%

\subsection{Determining When to Stop Splitting}

As discussed in Basic Ideas, in principal, splitting could continue until all cases are perfectly classified or predicted. However, this wouldn't make much sense since we would likely end up with a tree structure that is as complex and "tedious" as the original data file (with many nodes possibly containing single observations), and that would most likely not be very useful or accurate for predicting new observations. What is required is some reasonable stopping rule. In CART, two options are available that can be used to keep a check on the splitting process; namely Minimum n and Fraction of objects.

Minimum n. One way to control splitting is to allow splitting to continue until all terminal nodes are pure or contain no more than a specified minimum number of cases or objects. In CART this is done by using the option Minimum n that allows us to specify the desired minimum number of cases as a check on the splitting process. This option can be used when Prune on misclassification error, Prune on deviance, or Prune on variance is active as the Stopping rule for the analysis.

Fraction of objects. Another way to control splitting is to allow splitting to continue until all terminal nodes are pure or contain no more cases than a specified minimum fraction of the sizes of one or more classes (in the case of classification problems, or all cases in regression problems). This option can be used when FACT-style direct stopping has been selected as the Stopping rule for the analysis. In CART, the desired minimum fraction can be specified as the Fraction of objects.

 For classification problems, if the priors used in the analysis are equal and class sizes are equal as well, then splitting will stop when all terminal nodes containing more than one class have no more cases than the specified fraction of the class sizes for one or more classes. Alternatively, if the priors used in the analysis are not equal, splitting will stop when all terminal nodes containing more than one class have no more cases than the specified fraction for one or more classes
%-------------------------------------------------------------%

\subsection{Pruning and Selecting the "Right-Sized" Tree}

The size of a tree in the classification and regression trees analysis is an important issue, since an unreasonably big tree can only make the interpretation of results more difficult. Some generalizations can be offered about what constitutes the "right-sized" tree. It should be sufficiently complex to account for the known facts, but at the same time it should be as simple as possible. It should exploit information that increases predictive accuracy and ignore information that does not. It should, if possible, lead to greater understanding of the phenomena it describes.

 The options available in CART allow the use of either, or both, of two different strategies for selecting the "right-sized" tree from among all the possible trees. One strategy is to grow the tree to just the right size, where the right size is determined by the user, based on the knowledge from previous research, diagnostic information from previous analyses, or even intuition. The other strategy is to use a set of well-documented, structured procedures developed by Breiman et al. (1984) for selecting the "right-sized" tree. These procedures are not foolproof, but at least they take subjective judgment out of the process of selecting the "right-sized" tree.

FACT-style direct stopping. We will begin by describing the first strategy, in which the user specifies the size to grow the tree. This strategy is followed by selecting FACT-style direct stopping as the stopping rule for the analysis, and by specifying the Fraction of objects that allows the tree to grow to the desired size. CART provides several options for obtaining diagnostic information to determine the reasonableness of the choice of size for the tree. Specifically, three options are available for performing cross-validation of the selected tree; namely Test sample, V-fold, and Minimal cost-complexity.

Test sample cross-validation. 

\subsection{Test sample cross-validation}
The first, and most preferred type of cross-validation is the test sample cross-validation. In this type of cross-validation, the tree is computed from the learning sample, and its predictive accuracy is tested by applying it to predict the class membership in the test sample. If the costs for the test sample exceed the costs for the learning sample, then this is an indication of poor cross-validation. In that case, a different sized tree might cross-validate better. The test and learning samples can be formed by collecting two independent data sets, or if a large learning sample is available, by reserving a randomly selected proportion of the cases, say a third or a half, for use as the test sample.

Test sample cross-validation is performed by specifying a sample identifier variable that contains codes for identifying the sample (learning or test) to which each case or object belongs.


\subsubsection{V-fold cross-validation} 
The second type of cross-validation available in CART is V-fold cross-validation. This type of cross-validation is useful when no test sample is available and the learning sample is too small to have the test sample taken from it. The user-specified 'v' value for v-fold cross-validation (its default value is 3) determines the number of random subsamples, as equal in size as possible, that are formed from the learning sample. 

A tree of the specified size is computed 'v' times, each time leaving out one of the subsamples from the computations, and using that subsample as a test sample for cross-validation, so that each subsample is used (v - 1) times in the learning sample and just once as the test sample. The CV costs (cross-validation cost) computed for each of the 'v' test samples are then averaged to give the v-fold estimate of the CV costs.

\subsubsection{Minimal cost-complexity cross-validation pruning} 
In CART, minimal cost-complexity cross-validation pruning is performed, if Prune on misclassification error has been selected as the Stopping rule. On the other hand, if Prune on deviance has been selected as the Stopping rule, then minimal deviance-complexity cross-validation pruning is performed. The only difference in the two options is the measure of prediction error that is used. Prune on misclassification error uses the costs that equals the misclassification rate when priors are estimated and misclassification costs are equal, while Prune on deviance uses a measure, based on maximum-likelihood principles, called the deviance .
For details about the algorithms used in CART to implement Minimal cost-complexity cross-validation pruning, see also the Introductory Overview and Computational Methods sections of Classification Trees Analysis.

The sequence of trees obtained by this algorithm have a number of interesting properties. They are nested, because the successively pruned trees contain all the nodes of the next smaller tree in the sequence. Initially, many nodes are often pruned going from one tree to the next smaller tree in the sequence, but fewer nodes tend to be pruned as the root node is approached. The sequence of largest trees is also optimally pruned, because for every size of tree in the sequence, there is no other tree of the same size with lower costs. 
%Proofs and/or explanations of these properties can be found in Breiman et al. (1984).

\subsubsection{Tree selection after pruning}
The pruning, as discussed above, often results in a sequence of optimally pruned trees. So the next task is to use an appropriate criterion to select the "right-sized" tree from this set of optimal trees. A natural criterion would be the CV costs (cross-validation costs). While there is nothing wrong with choosing the tree with the minimum CV costs as the "right-sized" tree, oftentimes there will be several trees with CV costs close to the minimum. Following Breiman et al. (1984) we could use the "automatic" tree selection procedure and choose as the "right-sized" tree the smallest-sized (least complex) tree whose CV costs do not differ appreciably from the minimum CV costs. In particular, they proposed a "1 SE rule" for making this selection, i.e., choose as the "right-sized" tree the smallest-sized tree whose CV costs do not exceed the minimum CV costs plus 1 times the standard error of the CV costs for the minimum CV costs tree. In CART, a multiple other than the 1 (the default) can also be specified for the SE rule. Thus, specifying a value of 0.0 would result in the minimal CV cost tree being selected as the "right-sized" tree. Values greater than 1.0 could lead to trees much smaller than the minimal CV cost tree being selected as the "right-sized" tree. One distinct advantage of the "automatic" tree selection procedure is that it helps to avoid "over fitting" and "under fitting" of the data.

As can be been seen, minimal cost-complexity cross-validation pruning and subsequent "right-sized" tree selection is a truly "automatic" process. The algorithms make all the decisions leading to the selection of the "right-sized" tree, except for, perhaps, specification of a value for the SE rule. V-fold cross-validation allows us to evaluate how well each tree "performs" when repeatedly cross-validated in different samples randomly drawn from the data.


\subsection{Computational Formulas}

In Classification and Regression Trees, estimates of accuracy are computed by different formulas for categorical and continuous dependent variables (classification and regression-type problems). For classification-type problems (categorical dependent variable) accuracy is measured in terms of the true classification rate of the classifier, while in the case of regression (continuous dependent variable) accuracy is measured in terms of mean squared error of the predictor.

In addition to measuring accuracy, the following measures of node impurity are used for classification problems: The Gini measure, generalized Chi-square measure, and generalized G-square measure. The Chi-square measure is similar to the standard Chi-square value computed for the expected and observed classifications (with priors adjusted for misclassification cost), and the G-square measure is similar to the maximum-likelihood Chi-square (as for example computed in the Log-Linear module). The Gini measure is the one most often used for measuring purity in the context of classification problems, and it is described below.

\end{document}
