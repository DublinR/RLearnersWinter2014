%=========================================================================== %
\newpage
\section{Singular Value Decomposition (SVD)}
\textit{Source: : http://iridl.ldeo.columbia.edu/dochelp/StatTutorial/SVD/ \\}
Singular value decomposition (SVD) is quite possibly the most widely-used multivariate statistical technique used in the atmospheric sciences. The technique was first introduced to meteorology in a 1956 paper by Edward Lorenz, in which he referred to the process as empirical orthogonal function (EOF) analysis. Today, it is also commonly known as principal-component analysis (PCA). All three names are still used, and refer to the same set of procedures within the Data Library. 


The purpose of singular value decomposition is to reduce a dataset containing a large number of values to a dataset containing significantly fewer values, but which still contains a large fraction of the variability present in the original data. Often in the atmospheric and geophysical sciences, data will exhibit large spatial correlations. SVD analysis results in a more compact representation of these correlations, especially with multivariate datasets and can provide insight into spatial and temporal variations exhibited in the fields of data being analyzed. 


There are a few caveats one should be aware of before computing the SVD of a set of data. First, the data must consist of anomalies. Secondly, the data should be de-trended. When trends in the data exist over time, the first structure often captures them. If the purpose of the analysis is to find spatial correlations independent of trends, the data should be de-trended before applying SVD analysis. 

\newpage

\subsection*{Question 5}
Load the hand-written digits data using the following commands:

\begin{framed}
	\begin{verbatim}
	library(ElemStatLearn)
	data(zip.train)
	\end{verbatim}
\end{framed}

Each row of the `\texttt{zip.train}` data set corresponds to a hand written digit. The
first column of the zip.train data is the actual digit. The next 256 columns
are the intensity values for an image of the digit. To visualize the digit we
can use the `\texttt{zip2image()}` function to convert a row into a 16 x 16 matrix:

\begin{framed}
	\begin{verbatim}
	# Create an image matrix for the 3rd row, which is a 4
	im = zip2image(zip.train,3)
	image(im)
	\end{verbatim}
\end{framed}


Using the `zip2image` file, create an image matrix for the 8th and 18th rows.
\begin{figure}
	\centering
	\includegraphics[width=0.7\linewidth]{./DAquiz3Graph5b}
	\caption{}
	\label{fig:DAquiz3Graph5}
\end{figure}
For each image matrix calculate the `\textit{\textbf{svd}}` of the matrix (with no scaling). 

\begin{itemize}
	\item[(i)] What is the percent variance explained by the \textbf{first} singular vector for the image
	from the 8th row? 
	\item[(ii)] What is the percent variance explained for the image from the
	18th row? (by the \textbf{first} singular vector)
	\item[(iii)] Why is the percent variance lower for the image from the 18th row?
\end{itemize}
\newpage
\begin{framed}
	\begin{verbatim}
	
	im8 <- zip2image(zip.train, 8)
	im18 <- zip2image(zip.train, 18)
	
	svd8 <- svd(im8)
	svd18 <- svd(im18)
	
	par(mfrow=c(2,2))
	plot(svd8$d^2/sum(svd8$d^2),
	xlab="Column",ylab="Percent of variance explained for Row 8",
	pch=19)
	
	plot(svd18$d^2/sum(svd18$d^2),
	xlab="Column",ylab="Percent of variance explained for Row 18",
	pch=19)
	
	image(im8)
	
	image(im18)
	
	\end{verbatim}
\end{framed}
\newpage
From the image - the second image is more complicated, so there are multiple patterns each explaining a large percentage of variance.
\begin{figure}[h!]
	\centering
	\includegraphics[width=1.0\linewidth]{./DAquiz3Graph5b}
	\caption{}
	\label{fig:DAquiz3Graph5}
\end{figure}

\end{document}
~~The first singular vector explains 98\% of the variance for row 8 and 48\% for row 18. 
The reason the first singular vector explains less variance for the 18th row is because the 8th row has higher average values.~~
**The first singular vector explains 98\% of the variance for row 8 and 48\% for row 18. 
The reason the first singular vector explains less variance for the 18th row is that the image is more complicated, so there are multiple patterns each explaining a large percentage of variance.**
