\documentclass[12pt, a4paper]{report}
\usepackage{natbib}
\usepackage{vmargin}
\usepackage{epsfig}
\usepackage{subfigure}
%\usepackage{amscd}
\usepackage{amssymb}
\usepackage{amsbsy}
\usepackage{amsthm,amsmath}
%\usepackage[dvips]{graphicx}
\bibliographystyle{chicago}
\renewcommand{\baselinestretch}{1.2}

% left top textwidth textheight headheight % headsep footheight footskip
\setmargins{3.0cm}{2.5cm}{15.5 cm}{23.5cm}{0.5cm}{0cm}{1cm}{1cm}

\pagenumbering{arabic}


\begin{document}
\author{Kevin O'Brien}
\title{Linear Mixed Effects Models}
\date{\today}
\maketitle

\tableofcontents \setcounter{tocdepth}{2}

\newpage


\subsection{Using Linear Mixed Effects Models} It is shown that
classical models can give different results to linear mixed
effects models , based on the same data. \citet{Demi} illustrates
this with a comparison of simple regression model of prices
against sales, with a mixed model, that takes groups of data into
account.


\subsection{Laird Ware Model} Linear mixed effects models (LME)
differs from the conventional linear model in that it has both
fixed effects and random effects regressors, and coefficients
thereof. Further to a paper published by Laird and Ware in $1982$,
it is conventional to formulate  an LME in matrix form as follows:

  \begin{displaymath}
      Y_{i} =X_{i}\beta + Z_{i}b_{i} + \epsilon_{i}
  \end{displaymath}
\begin{itemize}
\item $Y_{i}$ is the $n \times 1$ response vector \item $X_{i}$ is
the $n \times p$ Model matrix for fixed effects \item $\beta$ is
the $p \times 1$ vector of fixed effects coefficients \item
$Z_{i}$ is the $n \times q$ Model matrix for random effects \item
$b_{i}$ is the $q \times 1$ vector of random effects coefficients,
sometimes denoted as $u_{i}$ \item $\epsilon$ is the $n \times 1$
vector of observation errors
\end{itemize}

\subsection{Computation} When tackling linear mixed effects models
using the R language, a statistician can call upon the \emph{lme}
command found in the \emph{nlme} package.This command fits a LME
model to the data set using either Maximum Likelihood (ML) or
Restricted Maximum Likelihood (REML).
\\
\\
The first two arguments for \emph{lme} are \emph{fixed} and
\emph{data}, which give the model for the expected responses (i.e.
the fixed part of the model), and the data that the model should
be fitted from. The next argument is  \emph{random}, a one-sided
formula which describes the random effects, and the grouping
structure for the model. The  \emph{method} argument can specify
whether to use 'REML', the default setting, or 'ML'.

\newpage


\section{Linear Mixed Effects Models}
\subsubsection{Applications}
So-called mixed-effect models (or just mixed models) include
additional random-effect terms, and are often appropriate for
representing clustered, and therefore dependent, data � arising,
for example, when data are collected hierarchically, when
observations are taken on related individuals (such as siblings),
or when data are gathered over time on the same individuals.
\subsubsection{Mean Centering}
It is customary to center data prior to running LMM or HLM.
Centering means subtracting the mean, so means become zero. Two
main types of centering are group mean centering and grand mean
centering.
\subsubsection{Restricted Maximum Likelihood}
restricted (or residual) maximum likelihood (REML) is a method for
fitting linear mixed models. In contrast to conventional maximum
likelihood estimation, REML can produce unbiased estimates of
variance and covariance parameters.

\section{Grouped Data Sets}
In modern statistical analysis , data sets have very complex
structures, such as  clustered data, repeated data and
hierarchical data (henceforth referred to as grouped data).

Repeated data considers various observations periodically taken
from the same subjects. `Before and after' measurements, as used
in paired t tests, are a well known example of repeated
measurements. Clustered data is simply the grouping of
observations according to common characteristics. For example, an
study of pupils of a school would account for the fact that they
are grouped according to their classes.

Hierarchical structures organize data into a tree-like structure,
i.e. groups within groups. Using the previous example, the pupils
would be categorized according to their years (i.e the parent
group) and then their classes (i.e the child group). This can be
extended again to multiple schools, where each school would be the
parent group of each year.

An important feature of such data sets is that there is
correlation between observations within each of the groups
\citep{Faraway}. Observations in different groups may be
independent, but any assumption that these observations within the
same group are independent is inappropriate . Consequently
\citet{Demi} states that there is two sources of variations to be
considered, `within groups' and `between groups'.


\subsection{Classical Models}
\citet{Demi} discusses the inadequacy of `classical models' in
analysing such data types, with particular reference to the simple
linear model . The simple linear model is a well known statistical
methodology that describes the relationship between dependent
variables $Y$ and an independent predictor variable $X$. Where
$Y={y_{1},y_{2},..y_{k}..y_{n}}$ and
$X={x_{1},x_{2},..x_{k}..x_{n}}$, an intercept $\alpha$ and slope
$\beta$ are estimated such that the error terms associated with
each observation $y_{i}$ is minimised.

\begin{equation}
y_{k} = \alpha + \beta x_{k}+ \epsilon_{k}
\end{equation}
In classical statistics a typical assumption is that observations
are drawn from the same general population, are independent and
identically distributed \citep{Demi}. Consequently there is no way
to account for the grouped nature of data sets described
previously, and so there lies the possibility of observations
being treated as independent measurements. \citet[pg.3]{Demi}
gives a very informative example wherein a classical approach is
compared to an approach that does account for grouping. The
conclusion to be drawn from Demidenko's example is that failure to
account for grouping leads to an incorrect conclusion about the
data. The approach recommended is known as `mixed models' and
shall be introduced presently.

\newpage

\section{Fixed Effects and Random Effects Models}
Before proceeding to a description of mixed models, an
introduction to fixed effects and random effects models is
required. This section follows on from the discussion of
measurement error models in the last chapter.

%A model should be regarded as mathematical description of
%observations with respect to the effect of a number of factors,
%and error terms.
\subsection{Fixed Effects}
\citet{McCullSearle} gives an example of a study where the
observations , occurrences of skin tumours called basal cell
epithelioma, were classified according to `factors', i.e. the
gender, age and exposure to sunshine of the patients. Levels are
the individual classes of each of these factors (e.g. `Male' and
`Female' would be the levels of the factor `Gender'). The
scientific interest lies in examining the extent to which
different factor levels affect the variable of interest. The
effects of a level of a factor are one of two types ; fixed
effects and random effects. Fixed effects describe effects due to
a finite set of levels for a factor (i.e multichotomous factors).
The factors described in the skin tumour example are all fixed
effects factors. Fixed effects models are the cases where only
fixed effects are present, with the exception of random error
terms.

To demonstrate fixed effects model \citet{Searle} describes a
study wherein $24$ plants are divided into four groups of six, and
each group is subjected to its own treatment regime. Three
different fertilizers are used with three of the groups
(treatments $N$,$P$,$K$), while no fertilizer is used on the
fourth group, (i.e. it is a control group denoted as $C$).
\citet{Searle} constructs a model to describe the crop yield
resultant from the experiment.

\begin{equation}
y_{ij} = \mu + \alpha_{i} + e_{ij}
\end{equation}

where $y_{ij}$ is the $j$th plant (i.e. crop yield) on the $i$th
treatment, with $\mu$ as the mean yield, $\alpha_{i}$ is the
effect of each fertilizer treatments (i.e. fixed effects) and
$e_{ij}$ is the error term. The fixed effect for each observation
is an unknown constant that is to be determined from computing the
data.

 The $\mu$ term would not
necessarily be present in all formulations. Some authors, such as
\citet{Demi}, may use a single term as equivalent to the $\mu$ and
$\alpha$ terms. (It is customary to centre data prior to using
mixed effects methodologies. Centering means subtracting the mean
of the observations from each observed value, so mean of the
resultant values become zero.)

\subsection{Random Effects}
The random effects model describes the case where there is an
infinite number of levels in a factor. In other words the factor
is a random variable. \citet{Searle} demonstrates this with a
second example; a study of the maternal ability of mice. In this
example 4 female mice , all of the same breed, have 6 litters each
over a period of time. The weights $w_{ij}$ of each litter ($j$)
from each mouse ($i$) were taken to be the proxy for maternal
ability and is formulated as follows;
\begin{equation}
w_{ij} = \mu + \delta_{i} + e_{ij}
\end{equation}

As in the previous exmaple, $\alpha$ is the mean , $e_{ij}$ is the
error term and $\delta_{i}$ is a random effect due to each mouse.
Notably these four mice are considered as a sample of the overall
population of female mice of that breed, consequently an important
characteristic of random effects models is that the $\delta_{i}$
values are a random sample of all $\delta$ terms. Therefore these
random terms can be used for making inferences about populations.
\citep{McCullSearle}.

\subsection{Variance Components}
Each random effect has an associated `variance component' term.
This is a model parameter which quantifies random variation due to
that effect only. Therefore for every observation there are two
sources of variation, random variation and residual variation, and
can be expressed as follows $var(y_{ij})=\sigma^{2}_{p} +
\sigma^{2}$. These variations are known as the variance
components. This is an important difference with fixed effects
models, which is subject to residual variation ( i.e $\sigma^{2}$)
only \citep{BrownPrescott}.

In fixed effects models there is no covariance between any pair of
observations. \citet*{BrownPrescott} shows that, while there is no
covariance between observations from different subjects (i.e the
mice in Searle's example), there exists correlation between
observations from the same subject (i.e. litter weights from the
same mouse are correlated). In this case the covariance is the
subjects variance component (i.e. $\sigma^{2}_{p}$)

\subsection{More complex examples}
\citet{Searle} offers elaborations on both examples used so far.
In the case of the fixed effects model, the model can be amended
to take account for different varieties of each plants being
studied. (The groups of six plants are subdivided into three
variety types.) $y_{ijk}$ is the yield of the $k$th plant of the
$j$th variety in the $i$th treatment, and is described as follows;

\begin{equation}
y_{ijk} = \mu + \alpha_{i} + \beta_{j} + \gamma_{ij} + e_{ijk}
\end{equation}
This new formulation includes a fixed effect $\beta_{j}$ to
account for the variety type, and an `interaction effect'
$\gamma_{ij}$. An interaction effect describes the combined
effects of two or more variables on the observation. Similarly the
random effects example is elaborated to account for the effects of
three different technicians. Again there is a random effect
component $\tau$ to account for these technicians, and an
interaction effect $\theta$ to account for the combined effect of
the mice and the technicians.

\begin{equation}
w_{ijk} = \mu + \delta_{i} + \tau_{j} + \theta_{ij} + e_{ijk}
\end{equation}


\subsubsection{Likelihood Ratio Tests} The problem with REML for
model building is that the "likelihoods" obtained for different
fixed effects are not comparable. Hence it is not valid to compare
models with different fixed effects using a likelihood ratio test
or AIC when REML is used to estimate the model. Therefore models
derived using ML must be used instead.





\newpage


In simple examples $V^{-1}$ is a straightforward calculation, but
with higher dimensions it becomes a very complex calculation.
\citet{Henderson50, Henderson63, Henderson73, Henderson84a}
derived the `Mixed Model Equations (MME)' to provide estimates for
$\beta$ and $u$ without the need to calculate $V^{-1}$.

\begin{equation}
\left( \begin{matrix}
X^{T}R^{-1}X  & X^{T}R^{-1}Z \\
Z^{T}R^{-1}X  &  Z^{T}R^{-1}Z + G^{-1}
 \end{matrix}\right)  \left(
\begin{array}{c}
\hat{\beta}  \\
\hat{u}\end{array} \right) = \left(  \begin{array}{c}
X^{T}R^{-1}y  \\
Z^{T}R^{-1}y  \end{array} \right).
\end{equation}

When $R$ and $G$ are diagonal, $R^{-1}$ and $G^{-1}$ are trivial
calculations, and therefore the matrices in the equation CC are
much simpler to calculate than using $V^{-1}$.

Each of the elements of the above matrices are submatrices.
$X^{T}R^{-1}X$ is a $p \times p$ matrix, $Z^{T}R^{-1}Z + G^{-1}$
is a $q \times q$ matrix. The remaining elements, which are
transposes of each other, are of dimensions $p \times q$ and $q
\times p$ respectively. Therefore the overall matrix is of
dimension $(p+q) \times (p+q)$. These dimensions are notably
smaller than $n \times n$, which would have been the case if
$V^{-1}$, and therefore the inversion is easier to compute.

Rearranging the equation CC, the BLUE of $\beta$, and the BLUP of
$u$ can be shown to be;

\begin{equation}
\hat{\beta} = (X^{T}V^{-1}X)^{-1}X^{T}V^{-1}y
\end{equation}
\begin{equation}
\hat{u} = GZ^{T}V^{-1}(y-X\hat{\beta})
\end{equation}

\newpage

\section{Model Selection} The previous section on estimation assumes
the specification of a mixed model in terms of X, Z, G, and R.
Even though $X$ and $Z$ have known elements, there is some
flexibility is specifying the form and construction is flexible,
and for a particular data set, there are numerous possibilities
that can be considered. Similarly, various potential covariance
structures for G and R may be considered.

First, subject matter considerations and objectives are of great
importance when selecting a model; refer to Diggle (1988) and
Lindsey (1993).

Second, when the data themselves are looked to for guidance, many
of the graphical methods and diagnostics appropriate for the
general linear model extend to the mixed model setting as well
(Christensen, Pearson, and Johnson 1992).

Likelihood-based approachs to the mixed model allow the comparison
of candidate models. The most common of these are the likelihood
ratio test and Akaike's and Schwarz's information criteria
(Bozdogan 1987; Wolfinger 1993).


\chapter{Linear Mixed Effects Models}
%%%%%%%%%%%%%%%%%%%%%%%%%%%%%%%%%%%%%%%%%%%%%%%%%%%%%%%%%%%%%%%%%%%%%%%%%%%%%%%%%%%%%%%%%%%%%%%% Complex Data

\section{Grouped Data Sets}
In modern statistical analysis , data sets have very complex
structures, such as  clustered data, repeated data and
hierarchical data (henceforth referred to as grouped data).

Repeated data considers various observations periodically taken
from the same subjects. `Before and after' measurements, as used
in paired t tests, are a well known example of repeated
measurements. Clustered data is simply the grouping of
observations according to common characteristics. For example, an
study of pupils of a school would account for the fact that they
are grouped according to their classes.

Hierarchical structures organize data into a tree-like structure,
i.e. groups within groups. Using the previous example, the pupils
would be categorized according to their years (i.e the parent
group) and then their classes (i.e the child group). This can be
extended again to multiple schools, where each school would be the
parent group of each year.

An important feature of such data sets is that there is
correlation between observations within each of the groups
\citep{Faraway}. Observations in different groups may be
independent, but any assumption that these observations within the
same group are independent is inappropriate . Consequently
\citet{Demi} states that there is two sources of variations to be
considered, `within groups' and `between groups'.


\subsection{Classical Models}
\citet{Demi} discusses the inadequacy of `classical models' in
analysing such data types, with particular reference to the simple
linear model . The simple linear model is a well known statistical
methodology that describes the relationship between dependent
variables $Y$ and an independent predictor variable $X$. Where
$Y={y_{1},y_{2},..y_{k}..y_{n}}$ and
$X={x_{1},x_{2},..x_{k}..x_{n}}$, an intercept $\alpha$ and slope
$\beta$ are estimated such that the error terms associated with
each observation $y_{i}$ is minimised.

\begin{equation}
y_{k} = \alpha + \beta x_{k}+ \epsilon_{k}
\end{equation}
In classical statistics a typical assumption is that observations
are drawn from the same general population, are independent and
identically distributed \citep{Demi}. Consequently there is no way
to account for the grouped nature of data sets described
previously, and so there lies the possibility of observations
being treated as independent measurements. \citet[pg.3]{Demi}
gives a very informative example wherein a classical approach is
compared to an approach that does account for grouping. The
conclusion to be drawn from Demidenko's example is that failure to
account for grouping leads to an incorrect conclusion about the
data. The approach recommended is known as `mixed models' and
shall be introduced presently.

\newpage

\section{Fixed Effects and Random Effects Models}
Before proceeding to a description of mixed models, an
introduction to fixed effects and random effects models is
required. This section follows on from the discussion of
measurement error models in the last chapter.

%A model should be regarded as mathematical description of
%observations with respect to the effect of a number of factors,
%and error terms.
\subsection{Fixed Effects}
\citet{McCullSearle} gives an example of a study where the
observations , occurrences of skin tumours called basal cell
epithelioma, were classified according to `factors', i.e. the
gender, age and exposure to sunshine of the patients. Levels are
the individual classes of each of these factors (e.g. `Male' and
`Female' would be the levels of the factor `Gender'). The
scientific interest lies in examining the extent to which
different factor levels affect the variable of interest. The
effects of a level of a factor are one of two types ; fixed
effects and random effects. Fixed effects describe effects due to
a finite set of levels for a factor (i.e multichotomous factors).
The factors described in the skin tumour example are all fixed
effects factors. Fixed effects models are the cases where only
fixed effects are present, with the exception of random error
terms.

To demonstrate fixed effects model \citet{Searle} describes a
study wherein $24$ plants are divided into four groups of six, and
each group is subjected to its own treatment regime. Three
different fertilizers are used with three of the groups
(treatments $N$,$P$,$K$), while no fertilizer is used on the
fourth group, (i.e. it is a control group denoted as $C$).
\citet{Searle} constructs a model to describe the crop yield
resultant from the experiment.

\begin{equation}
y_{ij} = \mu + \alpha_{i} + e_{ij}
\end{equation}

where $y_{ij}$ is the $j$th plant (i.e. crop yield) on the $i$th
treatment, with $\mu$ as the mean yield, $\alpha_{i}$ is the
effect of each fertilizer treatments (i.e. fixed effects) and
$e_{ij}$ is the error term. The fixed effect for each observation
is an unknown constant that is to be determined from computing the
data.

 The $\mu$ term would not
necessarily be present in all formulations. Some authors, such as
\citet{Demi}, may use a single term as equivalent to the $\mu$ and
$\alpha$ terms. (It is customary to centre data prior to using
mixed effects methodologies. Centering means subtracting the mean
of the observations from each observed value, so mean of the
resultant values become zero.)

\subsection{Random Effects}
The random effects model describes the case where there is an
infinite number of levels in a factor. In other words the factor
is a random variable. \citet{Searle} demonstrates this with a
second example; a study of the maternal ability of mice. In this
example 4 female mice , all of the same breed, have 6 litters each
over a period of time. The weights $w_{ij}$ of each litter ($j$)
from each mouse ($i$) were taken to be the proxy for maternal
ability and is formulated as follows;
\begin{equation}
w_{ij} = \mu + \delta_{i} + e_{ij}
\end{equation}

As in the previous exmaple, $\alpha$ is the mean , $e_{ij}$ is the
error term and $\delta_{i}$ is a random effect due to each mouse.
Notably these four mice are considered as a sample of the overall
population of female mice of that breed, consequently an important
characteristic of random effects models is that the $\delta_{i}$
values are a random sample of all $\delta$ terms. Therefore these
random terms can be used for making inferences about populations.
\citep{McCullSearle}.

\subsection{Variance Components}
Each random effect has an associated `variance component' term.
This is a model parameter which quantifies random variation due to
that effect only. Therefore for every observation there are two
sources of variation, random variation and residual variation, and
can be expressed as follows $var(y_{ij})=\sigma^{2}_{p} +
\sigma^{2}$. These variations are known as the variance
components. This is an important difference with fixed effects
models, which is subject to residual variation ( i.e $\sigma^{2}$)
only \citep{BrownPrescott}.

In fixed effects models there is no covariance between any pair of
observations. \citet*{BrownPrescott} shows that, while there is no
covariance between observations from different subjects (i.e the
mice in Searle's example), there exists correlation between
observations from the same subject (i.e. litter weights from the
same mouse are correlated). In this case the covariance is the
subjects variance component (i.e. $\sigma^{2}_{p}$)

\subsection{More complex examples}
\citet{Searle} offers elaborations on both examples used so far.
In the case of the fixed effects model, the model can be amended
to take account for different varieties of each plants being
studied. (The groups of six plants are subdivided into three
variety types.) $y_{ijk}$ is the yield of the $k$th plant of the
$j$th variety in the $i$th treatment, and is described as follows;

\begin{equation}
y_{ijk} = \mu + \alpha_{i} + \beta_{j} + \gamma_{ij} + e_{ijk}
\end{equation}
This new formulation includes a fixed effect $\beta_{j}$ to
account for the variety type, and an `interaction effect'
$\gamma_{ij}$. An interaction effect describes the combined
effects of two or more variables on the observation. Similarly the
random effects example is elaborated to account for the effects of
three different technicians. Again there is a random effect
component $\tau$ to account for these technicians, and an
interaction effect $\theta$ to account for the combined effect of
the mice and the technicians.

\begin{equation}
w_{ijk} = \mu + \delta_{i} + \tau_{j} + \theta_{ij} + e_{ijk}
\end{equation}

\section{Mixed Models}

%\citet{BrownPrescott} defines random effects as realizations of
%samples from a normal distribution with mean equal to zero.

All models are characterized by the mean $\alpha$ and the error
terms. In addition to these terms, any model described so far will
have either random effects terms or fixed effects terms and
accordingly are referred to as random or fixed models. Models that
have both fixed effects terms and random effects terms are known
as 'mixed effects models'. Once the theory underlying fixed and
random effects models has been fully understood, the progression
to understanding mixed models is very simple.

Elaborating on the original mice litter example, the six litters
by each mouse were fed according to three different dietary
treatments \citep{Searle}. Therefore a fixed effect $\phi_{j}$ has
been added to the model, which is now formulated as follows;
\begin{equation}
y_{ij} = \mu + \delta_{i} + \phi_{j} + \gamma_{ij} +
\epsilon_{ijk}
\end{equation}
As before, an interaction effect $\gamma_{ij}$ must also be added
to the model. In cases where the interaction term describes the
combined effect of fixed and random components, it should be
treated as random effect. The variance of the above model is
composed of the $\sigma^{2}_{\delta}$, $\sigma^{2}_{\gamma}$ and
$\sigma^{2}_{\epsilon}$ .


It may be shown that the interaction factors make no contribution
to the outcome, i.e $\gamma_{ij}$ is consistently calculated as
zero. Considering the skin tumour example, a person's age would
bear no relation to their gender and hence there would be
plausible interaction between the two factors. Indeed , in keeping
with the `Law of Parsimony', factors should be specified such that
each would convey separate information. However, interaction terms
are extant when the model specifies repeated observations, as
there is necessarily a relationship between observations from the
same subject. Importantly, interaction effects, being random
effects, are attended by variance component terms and therefore
also contribute to the overall variance of the model.

\citet{Searle} gives a mixed effects model formulation for the
Grubbs artillery study. $y_{ij}$ is the muzzle velocity of the
$i$th shell, as measured by the $j$th chronometer.
\begin{equation}
y_{ij} = \mu + \alpha_{i} + \beta_{j}  + \epsilon_{ij}
\end{equation}
In this formulation $\alpha_{i}$ is the random effect of round
$i$, and the fixed effect component $\beta_{j}$ is the bias in
chronometer $j$. (Also, no interaction term is used).

\subsection{Fixed or Random?}

In the examples discussed so far, it is clear when effects are
fixed or random. In general, however, the difference is not as
clear. \citet{Searle} discusses the decision whether an effect
should be treated as random or fixed, stating that it depends upon
the context of the study, and how the data was gathered. \emph{The
situation to which the model applies is the deciding factor in
determining whether effects are fixed or random}.

Referring to the Grubbs data, the shells fired are a random sample
of shells therefore $\alpha_{i}$ components should be considered
random effects. Conversely $\beta_{j}$ are fixed effects
components because the three measurement devices are the only
instruments of interest \citep{Searle}.






\subsection{Estimators and Predictors}

The best linear unbiased predictor (BLUP) is used to estimating
random effects, i.e to derive \textbf{u}. The best linear unbiased
estimator (BLUE) is used to estimate the fixed effects,
\textbf{b}. They were formulated in a paper by \cite{Henderson59},
which provides the derivations of both. Inferences about fixed
effects have come tobe called `estimates', whereas inferences
about random effects have come be called `predictions`. hence the
naming of BLUP is to reinforce distinction between the two , but
it is essentially the same principal involved in both cases,
\citep{Robinson}. The procedures are known as the `best' in the
sense that they minimise the sampling variance and unbiased in the
sense that E[BLUE(\textbf{b})]= \textbf{b} and E[BLUP(\textbf{u})]
= \textbf{u}. The BLUE of \textbf{b}, and the BLUP of \textbf{u}
can be shown to be;

\begin{equation}
\hat{b} = (X^{T}V^{-1}X)^{-1}X^{T}V^{-1}y
\end{equation}
\begin{equation}
\hat{u} = GZ^{T}V^{-1}(y-X\hat{b})
\end{equation}

The practical application of both expressions requires that the
variance components be known. Therefore an estimate for the
variance components must be derived to analysis by either ANOVA,
or REML, a method that shall be introduced shortly. Importantly
calculations based on the above formulae require the calculation
of the inverse of \textbf{V}. In simple examples $V^{-1}$ is a
straightforward calculation, but with higher dimensions it becomes
a very complex calculation.



\subsection{Henderson's Mixed Model Equations}
\citet{Henderson50, Henderson63, Henderson73,
Henderson84a} derived the `mixed model equations (MME)' to provide
estimates for \textbf{b} and \textbf{u}without the need to
calculate the inverse of \textbf{V}.

\begin{equation}
\left(\begin{matrix}
X^{T}R^{-1}X  & X^{T}R^{-1}Z \\
Z^{T}R^{-1}X  & Z^{T}R^{-1}Z + G^{-1}
\end{matrix}\right) \left(\begin{array}{c}
\hat{\beta}  \\
\hat{u}\end{array} \right) = \left(  \begin{array}{c}
X^{T}R^{-1}y  \\
Z^{T}R^{-1}y  \end{array} \right).
\end{equation}

When $\textbf{R}$ and $\textbf{G}$  are diagonal, determining the
inverses thereof are trivial calculations, and therefore the above
matrices are much simpler to solve , and overcomes the problem
posed by the inverse of \textbf{V}.

Each of the elements of the above matrices are submatrices.
$X^{T}R^{-1}X$ is a $p \times p$ matrix, $Z^{T}R^{-1}Z + G^{-1}$
is a $q \times q$ matrix. The remaining elements, which are
transposes of each other, are of dimensions $p \times q$ and $q
\times p$ respectively. Therefore the overall matrix is of
dimension $(p+q) \times (p+q)$. These dimensions are notably
smaller than $n \times n$, which would have been the case if
$V^{-1}$, and therefore the inversion is easier to compute.

\section{Estimability of Fixed Effects}
Potentially it may be impossible to compute unique BLUE estimates for all the fixed factors in a model. This may be due to linear dependence in the model
matrix \textbf{X}. Consider the following example;


\begin{equation}
\textbf{b}= \left( \begin{array}{c}
  b_{1} \\
  b_{2} \\
  b_{3} \\
\end{array}  \right)
\end{equation}

\textbf{HERE}




\subsubsection{Likelihood Ratio Tests} The problem with REML for
model building is that the "likelihoods" obtained for different
fixed effects are not comparable. Hence it is not valid to compare
models with different fixed effects using a likelihood ratio test
or AIC when REML is used to estimate the model. Therefore models
derived using ML must be used instead.



%%%%%%%%%%%%%%%%%%%%%%%%%%%%%%%%%%%%%%%%%%%%%%%%%%%%%%%%%%%%%%%%%%%%%%%%%%%%%%%%%%%%%%%%%%%%%%%%%%%


%%%%%%%%%%%%%%%%%%%%%%%%%%%%%%%%%%%%%%%%%%%%%%%%%%%%%%%%%%%%%%%%%%%%%%%%%%%%%%%%%%%%%%%%%%%%%%%%%%% COMPUTATION WITH R
\section{Computation of LMEs using R} \cite{PB} advises how to
implement LME models in statistical software (ostensibly for S and
S PLUS, but R is very similar). When tackling linear mixed effects
models using the R language, a statistician can call upon the
\emph{lme} command found in the \emph{nlme} package.This command
fits a LME model to the data set using either Maximum Likelihood
(ML) or Restricted Maximum Likelihood (REML). ML may be referred
to as 'full maximum likelihood' estimation.

The first two arguments for \emph{lme} are \emph{fixed} and
\emph{data}, which give the model for the expected responses (i.e.
the fixed part of the model), and the data that themodel should be
fitted from. The next argument is  \emph{random}, a one-sided
formula which describes the random effects, and the grouping
structure for the model. The  \emph{method} argument can specify
whether to use 'REML', the default setting, or 'ML'.

\citet{PB} describes an experiment whereby the productivity of six
randomly chosen workers are assessed three times on each of three
machines, yielding the 54 observations in the following table.

\newpage
% latex table generated in R 2.9.2 by xtable 1.5-5 package
% Wed Sep 16 13:56:04 2009
\begin{table}[h!]
\begin{center}
\begin{tabular}{|c|c|c|c||c|c|c|c|}
  \hline
Observation & Worker & Machine & score & Observation & Worker & Machine & score \\
  \hline
1 & 1 & A & 52.00 &	28 & 4 & B & 63.20 \\
  2 & 1 & A & 52.80 &	  29 & 4 & B & 62.80 \\
  3 & 1 & A & 53.10 &	  30 & 4 & B & 62.20 \\
  4 & 2 & A & 51.80 &	  31 & 5 & B & 64.80 \\
  5 & 2 & A & 52.80 &	  32 & 5 & B & 65.00 \\
  6 & 2 & A & 53.10 &	  33 & 5 & B & 65.40 \\
  7 & 3 & A & 60.00 &	  34 & 6 & B & 43.70 \\
  8 & 3 & A & 60.20 &	  35 & 6 & B & 44.20 \\
  9 & 3 & A & 58.40 &	  36 & 6 & B & 43.00 \\
  10 & 4 & A & 51.10 &	  37 & 1 & C & 67.50 \\
  11 & 4 & A & 52.30 &	  38 & 1 & C & 67.20 \\
  12 & 4 & A & 50.30 &	  39 & 1 & C & 66.90 \\
  13 & 5 & A & 50.90 &	  40 & 2 & C & 61.50 \\
  14 & 5 & A & 51.80 &	  41 & 2 & C & 61.70 \\
  15 & 5 & A & 51.40 &	  42 & 2 & C & 62.30 \\
  16 & 6 & A & 46.40 &	  43 & 3 & C & 70.80 \\
  17 & 6 & A & 44.80 &	  44 & 3 & C & 70.60 \\
  18 & 6 & A & 49.20 &	  45 & 3 & C & 71.00 \\
  19 & 1 & B & 62.10 &	  46 & 4 & C & 64.10 \\
  20 & 1 & B & 62.60 &	  47 & 4 & C & 66.20 \\
  21 & 1 & B & 64.00 &	  48 & 4 & C & 64.00 \\
  22 & 2 & B & 59.70 &	  49 & 5 & C & 72.10 \\
  23 & 2 & B & 60.00 &	  50 & 5 & C & 72.00 \\
  24 & 2 & B & 59.00 &	  51 & 5 & C & 71.10 \\
  25 & 3 & B & 68.60 &	  52 & 6 & C & 62.00 \\
  26 & 3 & B & 65.80 &	  53 & 6 & C & 61.40 \\
  27 & 3 & B & 69.70 &	  54 & 6 & C & 60.50 \\

   \hline
\end{tabular}
\caption{Machines Data , Pinheiro Bates}
\end{center}
\end{table}
(Overall mean score = $59.65$, mean on machine A = $52.35$ , mean
on machine B = $60.32$, mean on machine C = $66.27$)
\newpage

The `worker' factor is modelled with random effects($u_{i}$),
whereas the `machine' factor is modelled with fixed effects
($\beta_{j}$). Due to the repeated nature of the data, interaction
effects between these factors are assumed to be extant, and shall
be examined accordingly. The interaction effect in this case
($\tau_{ij}$) describes whether the effect of changing from one
machine to another is different for each worker. The productivity
score $y_{ijk}$ is the $k$th observation taken on worker $i$ on
machine $j$, and is formulated
 as follows;

\begin{equation}
y_{ijk} = \beta_{j} + u_{i} + \tau_{ij} + \epsilon_{ijk}
\end{equation}
\begin{equation*}
u_{i} \sim N(0, \sigma^{2}_{u}), \epsilon_{ijk} \sim N(0,
\sigma^{2}), \tau_{i} \sim N(0, \sigma^{2}_{\tau})
\end{equation*}

The `nlme' package is incorporated into the R programming to
perform linear mixed model calculations. For the `Machines' data,
\citet{pb} use the following code, with the hierarchical structure
specified in the last argument.
\begin{verbatim}
lme(score~Machine, data=Machines, random=~1|Worker/Machine)
\end{verbatim}


The output of the R computation is given below.
\begin{verbatim}
Linear mixed-effects model fit by REML
  Data: Machines
  Log-restricted-likelihood: -107.8438
  Fixed: score ~ Machine
(Intercept)    MachineB    MachineC
  52.355556    7.966667   13.916667

Random effects:
 Formula: ~1 | Worker
        (Intercept)
StdDev:     4.78105

 Formula: ~1 | Machine %in% Worker
        (Intercept)  Residual
StdDev:    3.729532 0.9615771

Number of Observations: 54 Number of Groups:
             Worker Machine %in% Worker
                  6                  18

\end{verbatim}

\newpage
The crucial pieces of information given in the programme output
are the estimates of the intercepts for each of the three
machines. Machine A, which is treated as a control case, is
estimated to have an intercept of 52.35. The intercept estimates
for machines B and C are found to be $60.32$ and $66.27$ (by
adding the values 7.96 and 13.91 to 52.35 respectively). Estimate
for the variance components are also given; $\sigma^{2}_{u} =
(4.78)^{2}$ , $\sigma^{2}_{\tau} = (3.73)^{2}$ and
$\sigma^{2}_{\epsilon} = (0.96)^{2}$.

%%%%%%%%%%%%%%%%%%%%%%%%%%%%%%%%%%%%%%%%%%%%%%%%%%%%%%%%%%%%%%%%%%%%%%%%%%%%%%%%%%%%%%%%%%%%%%%%%%% MODEL SELECTION

\section{Model Selection}
\subsection{Akaike Information Critierion}
This is a model selection method, assessing how the goodness of
fit of a model. It is computed as follows:
\begin{displaymath}
      AIC = -2l_{max}+ 2k
\end{displaymath}
with $l_{max}$ as the log-likelihood maximum and $k$ as the number
of parameters. The candidate model with the lowest AIC value is
considered the best fitting of the candidate models.

\citet[p.13]{Demi} reports that some researchers have noted that
there is a bias present in AIC estimation, and have proposed
alternative formulations to rectify it. It is also reported that
AIC doesn't address the issue of multicollinearity sufficiently.
\citet{Demi} formulate an adaption; the Healthy AIC. It is
constructed to overcome the issue of ill posed models. The HAIC
will choose the candidate model with the shortest parameter vector
length.
%%%%%%%%%%%%%%%%%%%%%%%%%%%%%%%%%%%%%%%%%%%%%%%%%%%%%%%%%%%%%%%%%%%%%%%%%%%%%%%%%%%%%%%%%%%%%%%%%%%%%%%%%%%%%%%%%%%%%%%%%%%%%%%

\chapter{Errata}
\section{Distinction between Classical Models and Mixed Models}
\citet{Demi} discusses the inadequacy of `classical models' in
analysing such data types, with particular reference to the simple
linear model. The simple linear model is a well known statistical
methodology that describes the relationship between dependent
variables $Y$ and an independent predictor variable $X$. Where
$Y={y_{1},y_{2},..y_{k}..y_{n}}$ and
$X={x_{1},x_{2},..x_{k}..x_{n}}$, an intercept $\alpha$ and slope
$\beta$ are estimated such that the error terms associated with
each observation $y_{i}$ is minimized.

\begin{equation}
y_{k} = \alpha + \beta x_{k}+ \epsilon_{k}
\end{equation}
In classical statistics a typical assumption is that observations
are drawn from the same general population, are independent and
identically distributed \citep{Demi}. Consequently there is no way
to account for the grouped nature of data sets described
previously, and so there lies the possibility of observations
being treated as independent measurements. \citet[pg.3]{Demi}
gives a very informative example wherein a classical approach is
compared to an approach that does account for grouping. The
conclusion to be drawn from Demidenko's example is that failure to
account for grouping leads to an incorrect conclusion about the
data. The approach recommended is known as `mixed models' and
shall be introduced presently.

\newpage
\section{Maximum Likelihood Estimation}
Maximum likelihood (ML) estimation is a well known method of
obtaining estimates of unknown parameters by optimizing a
likelihood function. Models fitted by ML estimation can be
compared using the likelihood ratio test. However ML is known to
underestimate variance components for finite samples \citep{Demi}.

\subsection{Restricted Likelihood Estimation}

A method related to ML is restricted maximum likelihood
estimation(REML). REML was developed by \citet*{PT71} and
\citet{Harville} to provide unbiased estimates of variance and
covariance parameters. REML obtains estimates of the fixed effects
using non-likelihoodlike methods, such as ordinary least squares
or generalized least squares, and then using these estimates it
maximizes the likelihood of the residuals (subtracting off the
fixed effects) to obtain estimates of the variance parameters. In
most software packages REML is the default algorithm used to
compute coefficients for the predictor variables. REML estimation
reduces the bias in the variance component, and also handles high
correlations more effectively, and is less sensitive to outliers
than ML.

\citet{McCullSearle} describes two important outcomes of using
REML. Firstly variance components can be estimated without being
affected by fixed effects. Secondly in estimating variance
components with REML, degrees of freedom for the fixed effects can
be taken into account implicitly, whereas with ML they are not.
When estimating variance from normally distributed data, the ML
estimator for $\sigma^{2}$ is $\frac{S_{yy}}{n}$ whereas the REML
estimator is $\frac{S_{yy}}{n-1}$. ( $S_{yy}$ is the sum of square
identity;
\begin{equation}
S_{yy} = \Sigma_{i=i}^{N} (y-\bar{y})^{2}
\end{equation}








\newpage

\section{Fixed Effects and Random Effects Models}
Before proceeding to a description of mixed models, an
introduction to fixed effects and random effects models is
required. This section follows on from the discussion of
measurement error models in the last chapter.

%A model should be regarded as mathematical description of
%observations with respect to the effect of a number of factors,
%and error terms.
\subsection{Fixed Effects}
\citet{McCullSearle} gives an example of a study where the
observations , occurrences of skin tumours called basal cell
epithelioma, were classified according to `factors', i.e. the
gender, age and exposure to sunshine of the patients. Levels are
the individual classes of each of these factors (e.g. `Male' and
`Female' would be the levels of the factor `Gender'). The
scientific interest lies in examining the extent to which
different factor levels affect the variable of interest. The
effects of a level of a factor are one of two types ; fixed
effects and random effects. Fixed effects describe effects due to
a finite set of levels for a factor (i.e multichotomous factors).
The factors described in the skin tumour example are all fixed
effects factors. Fixed effects models are the cases where only
fixed effects are present, with the exception of random error
terms.

To demonstrate fixed effects model \citet{Searle} describes a
study wherein $24$ plants are divided into four groups of six, and
each group is subjected to its own treatment regime. Three
different fertilizers are used with three of the groups
(treatments $N$,$P$,$K$), while no fertilizer is used on the
fourth group, (i.e. it is a control group denoted as $C$).
\citet{Searle} constructs a model to describe the crop yield
resultant from the experiment.

\begin{equation}
y_{ij} = \mu + \alpha_{i} + e_{ij}
\end{equation}

where $y_{ij}$ is the $j$th plant (i.e. crop yield) on the $i$th
treatment, with $\mu$ as the mean yield, $\alpha_{i}$ is the
effect of each fertilizer treatments (i.e. fixed effects) and
$e_{ij}$ is the error term. The fixed effect for each observation
is an unknown constant that is to be determined from computing the
data.

 The $\mu$ term would not
necessarily be present in all formulations. Some authors, such as
\citet{Demi}, may use a single term as equivalent to the $\mu$ and
$\alpha$ terms. (It is customary to centre data prior to using
mixed effects methodologies. Centering means subtracting the mean
of the observations from each observed value, so mean of the
resultant values become zero.)

\subsection{Random Effects}
The random effects model describes the case where there is an
infinite number of levels in a factor. In other words the factor
is a random variable. \citet{Searle} demonstrates this with a
second example; a study of the maternal ability of mice. In this
example 4 female mice , all of the same breed, have 6 litters each
over a period of time. The weights $w_{ij}$ of each litter ($j$)
from each mouse ($i$) were taken to be the proxy for maternal
ability and is formulated as follows;
\begin{equation}
w_{ij} = \mu + \delta_{i} + e_{ij}
\end{equation}

As in the previous exmaple, $\alpha$ is the mean , $e_{ij}$ is the
error term and $\delta_{i}$ is a random effect due to each mouse.
Notably these four mice are considered as a sample of the overall
population of female mice of that breed, consequently an important
characteristic of random effects models is that the $\delta_{i}$
values are a random sample of all $\delta$ terms. Therefore these
random terms can be used for making inferences about populations.
\citep{McCullSearle}.

\subsection{Variance Components}
Each random effect has an associated `variance component' term.
This is a model parameter which quantifies random variation due to
that effect only. Therefore for every observation there are two
sources of variation, random variation and residual variation, and
can be expressed as follows $var(y_{ij})=\sigma^{2}_{p} +
\sigma^{2}$. These variations are known as the variance
components. This is an important difference with fixed effects
models, which is subject to residual variation ( i.e $\sigma^{2}$)
only \citep{BrownPrescott}.

In fixed effects models there is no covariance between any pair of
observations. \citet*{BrownPrescott} shows that, while there is no
covariance between observations from different subjects (i.e the
mice in Searle's example), there exists correlation between
observations from the same subject (i.e. litter weights from the
same mouse are correlated). In this case the covariance is the
subjects variance component (i.e. $\sigma^{2}_{p}$)

\subsection{More complex examples}
\citet{Searle} offers elaborations on both examples used so far.
In the case of the fixed effects model, the model can be amended
to take account for different varieties of each plants being
studied. (The groups of six plants are subdivided into three
variety types.) $y_{ijk}$ is the yield of the $k$th plant of the
$j$th variety in the $i$th treatment, and is described as follows;

\begin{equation}
y_{ijk} = \mu + \alpha_{i} + \beta_{j} + \gamma_{ij} + e_{ijk}
\end{equation}
This new formulation includes a fixed effect $\beta_{j}$ to
account for the variety type, and an `interaction effect'
$\gamma_{ij}$. An interaction effect describes the combined
effects of two or more variables on the observation. Similarly the
random effects example is elaborated to account for the effects of
three different technicians. Again there is a random effect
component $\tau$ to account for these technicians, and an
interaction effect $\theta$ to account for the combined effect of
the mice and the technicians.

\begin{equation}
w_{ijk} = \mu + \delta_{i} + \tau_{j} + \theta_{ij} + e_{ijk}
\end{equation}

\section{Mixed Models}

%\citet{BrownPrescott} defines random effects as realizations of
%samples from a normal distribution with mean equal to zero.

All models are characterized by the mean $\alpha$ and the error
terms. In addition to these terms, any model described so far will
have either random effects terms or fixed effects terms and
accordingly are referred to as random or fixed models. Models that
have both fixed effects terms and random effects terms are known
as 'mixed effects models'. Once the theory underlying fixed and
random effects models has been fully understood, the progression
to understanding mixed models is very simple.

Elaborating on the original mice litter example, the six litters
by each mouse were fed according to three different dietary
treatments \citep{Searle}. Therefore a fixed effect $\phi_{j}$ has
been added to the model, which is now formulated as follows;
\begin{equation}
y_{ij} = \mu + \delta_{i} + \phi_{j} + \gamma_{ij} +
\epsilon_{ijk}
\end{equation}
As before, an interaction effect $\gamma_{ij}$ must also be added
to the model. In cases where the interaction term describes the
combined effect of fixed and random components, it should be
treated as random effect. The variance of the above model is
composed of the $\sigma^{2}_{\delta}$, $\sigma^{2}_{\gamma}$ and
$\sigma^{2}_{\epsilon}$ .


It may be shown that the interaction factors make no contribution
to the outcome, i.e $\gamma_{ij}$ is consistently calculated as
zero. Considering the skin tumour example, a person's age would
bear no relation to their gender and hence there would be
plausible interaction between the two factors. Indeed , in keeping
with the `Law of Parsimony', factors should be specified such that
each would convey separate information. However, interaction terms
are extant when the model specifies repeated observations, as
there is necessarily a relationship between observations from the
same subject. Importantly, interaction effects, being random
effects, are attended by variance component terms and therefore
also contribute to the overall variance of the model.

\citet{Searle} gives a mixed effects model formulation for the
Grubbs artillery study. $y_{ij}$ is the muzzle velocity of the
$i$th shell, as measured by the $j$th chronometer.
\begin{equation}
y_{ij} = \mu + \alpha_{i} + \beta_{j}  + \epsilon_{ij}
\end{equation}
In this formulation $\alpha_{i}$ is the random effect of round
$i$, and the fixed effect component $\beta_{j}$ is the bias in
chronometer $j$. (Also, no interaction term is used).

\subsection{Fixed or Random?}

In the examples discussed so far, it is clear when effects are
fixed or random. In general, however, the difference is not as
clear. \citet{Searle} discusses the decision whether an effect
should be treated as random or fixed, stating that it depends upon
the context of the study, and how the data was gathered. \emph{The
situation to which the model applies is the deciding factor in
determining whether effects are fixed or random}.

Referring to the Grubbs data, the shells fired are a random sample
of shells therefore $\alpha_{i}$ components should be considered
random effects. Conversely $\beta_{j}$ are fixed effects
components because the three measurement devices are the only
instruments of interest \citep{Searle}.


\subsection{Advantages of Mixed Models}
\citet{BrownPrescott} discusses the  following advantages of using
mixed effects models. In the case of repeated measurements , it is
appropriate to take account of the correlation of each group of
observations. Mixed models lead to more appropriate estimates and
standard errors for fixed effects, particularly in the case of
repeated measures. Analysis using a mixed model is more
appropriate for inference on a hierarchical data. In the case of
unbalanced data, mixed models are more appropriate than other
methodologies.

\citet{Demi} comments that mixed models are the correct approach
for dealing with grouped data. The use of linear mixed effects
models has advanced greatly with increased usage of statistical
software. This author also notes that mixed models are a hybrid of
bayesian and frequentist methodologies and that mixed model
approaches are more flexible than bayesian.


\subsubsection{Unbalanced Data} Unbalanced data refers to situations where these groups are
of different sizes. Mixed Effects Models are suitable for studying
unbalanced data sets. The variance components of random effects
for these set can not be derived using alternative methods such as
ANOVA.

\subsection{Matrix Formulation} There are matrix (i.e multivariate)
formulations of both fixed effects models and random effects
models. \citet{BrownPrescott} remarks that the matrix notation
makes the underlying theory of mixed effects models much easier to
work with. The fixed effects models can be specified as follows;

\begin{equation}
\textbf{Y} = \textbf{Xb} + \textbf{e}
\end{equation}

\textbf{Y} is the vector of $n$ observations, with dimension $n
\times 1$. \textbf{b} is a vector of fixed $p$ effects, and has
dimension $p \times 1$. It is composed of coefficients, with the
first element being the population mean. For the skin tumour
example, with the three specified fixed effects, $p=4$. \textbf{X}
is known as the design `matrix', model matrix for fixed effects,
and comprises $0$s or $1$s, depending on whether the relevant
fixed effects have any effect on the observation is question.
\textbf{X} has dimension $n \times p$. \textbf{e} is the vector of
residuals with dimension $n \times 1$.

The random effects models can be specified similarly. \textbf{Z}
is known as the `model matrix for random effects', and also
comprises $0$s or $1$s. It has dimension $n \times q$. \textbf{u}
is a vector of random $q$ effects, and has dimension $q \times 1$.

\begin{equation}
\textbf{Y} = \textbf{Zu} + \textbf{e}
\end{equation}

Again, once the component fixed effects and random effects
components are considered, progression to a mixed model
formulation is a simple step. Further to \citet{lw82}, it is
conventional to formulate a mixed effects model in matrix form as
follows:

\begin{equation}
\textbf{Y} = \textbf{Xb} + \textbf{Zu} + \textbf{e}
\end{equation}

($E(\textbf{u})=0$, $E(\textbf{e})=0 $ and $E(\textbf{y}) =
\textbf{Xb}$)

\section{Mixed Model Calculations}
\subsection{Formulation of the Variance Matrix V}
\textbf{V} , the variance matrix of \textbf{Y}, can be expressed
as follows;
\begin{eqnarray}
\textbf{V}= var ( \textbf{Xb} + \textbf{Zu} + \textbf{e})\\
\textbf{V}= var ( \textbf{Xb} ) + var (\textbf{Zu}) +
var(\textbf{e}))
\end{eqnarray}

%%%%%%%%%%%%%%%%%%%%%%%%%%%%%

$var(\textbf{Xb})$ is known to be zero. $var(\textbf{Zu})$ can be
written as $Zvar(\textbf{u})Z^{T}$. $Z$ is a matrix of constants.
By letting $var(u) = G$ (i.e $\textbf{u} ~ N(0,\textbf{G})$), this
becomes $ZGZ^{T}$. This specifies the covariance due to random
effects. The residual covariance matrix $var(e)$ is denoted as
$R$, ($\textbf{e} ~ N(0,\textbf{R})$). Residual are uncorrelated,
hence \textbf{R} is equivalent to $\sigma^{2}$\textbf{I}, where
\textbf{I} is the identity matrix. The variance matrix \textbf{V}
can therefore be written as;

\begin{equation}
\textbf{V}  = ZGZ^{T} + \textbf{R}
\end{equation}

\subsection{Estimators and Predictors}

The best linear unbiased predictor (BLUP) is used to estimating
random effects, i.e to derive \textbf{u}. The best linear unbiased
estimator (BLUE) is used to estimate the fixed effects,
\textbf{b}. They were formulated in a paper by \cite{Henderson59},
which provides the derivations of both. Inferences about fixed
effects have come tobe called `estimates', whereas inferences
about random effects have come be called `predictions`. hence the
naming of BLUP is to reinforce distinction between the two , but
it is essentially the same principal involved in both cases,
\citep{Robinson}. The procedures are known as the `best' in the
sense that they minimise the sampling variance and unbiased in the
sense that E[BLUE(\textbf{b})]= \textbf{b} and E[BLUP(\textbf{u})]
= \textbf{u}. The BLUE of \textbf{b}, and the BLUP of \textbf{u}
can be shown to be;

\begin{equation}
\hat{b} = (X^{T}V^{-1}X)^{-1}X^{T}V^{-1}y
\end{equation}
\begin{equation}
\hat{u} = GZ^{T}V^{-1}(y-X\hat{b})
\end{equation}

The practical application of both expressions requires that the
variance components be known. Therefore an estimate for the
variance components must be derived to analysis by either ANOVA,
or REML, a method that shall be introduced shortly. Importantly
calculations based on the above formulae require the calculation
of the inverse of \textbf{V}. In simple examples $V^{-1}$ is a
straightforward calculation, but with higher dimensions it becomes
a very complex calculation.



\subsection{Henderson's Mixed Model Equations}
\citet{Henderson50, Henderson63, Henderson73, Henderson84a}
derived the `mixed model equations (MME)' to provide estimates for
\textbf{b} and \textbf{u}without the need to calculate the inverse
of \textbf{V}.

\begin{equation}
\left(\begin{matrix}
X^{T}R^{-1}X  & X^{T}R^{-1}Z \\
Z^{T}R^{-1}X  & Z^{T}R^{-1}Z + G^{-1}
\end{matrix}\right) \left(\begin{array}{c}
\hat{\beta}  \\
\hat{u}\end{array} \right) = \left(  \begin{array}{c}
X^{T}R^{-1}y  \\
Z^{T}R^{-1}y  \end{array} \right).
\end{equation}

When $\textbf{R}$ and $\textbf{G}$  are diagonal, determining the
inverses thereof are trivial calculations, and therefore the above
matrices are much simpler to solve , and overcomes the problem
posed by the inverse of \textbf{V}.

Each of the elements of the above matrices are submatrices.
$X^{T}R^{-1}X$ is a $p \times p$ matrix, $Z^{T}R^{-1}Z + G^{-1}$
is a $q \times q$ matrix. The remaining elements, which are
transposes of each other, are of dimensions $p \times q$ and $q
\times p$ respectively. Therefore the overall matrix is of
dimension $(p+q) \times (p+q)$. These dimensions are notably
smaller than $n \times n$, which would have been the case if
$V^{-1}$, and therefore the inversion is easier to compute.


%%%%%%%%%%%%%%%%%%%%%%%%%%%%%%%%%%%%%%%%%%%%%%%%%%%%%%%%%%%%%%%%%%%%%%%%%%%%%%%% ML and REML


%%%%%%%%%%%%%%%%%%%%%%%%%%%%%%%%%%%%%%%%%%%%%%%%%%%%%%%%%%%%%%%%%%%%%%%%%%%%%%%%%%%%%%%%%%%%%%%%%%




%%%%%%%%%%%%%%%%%%%%%%%%%%%%%%%%%%%%%%%%%%%%%%%%%%%%%%%%%%%%%%%%%%%%%%%%%%%%%%%%%%%%%%%%%%%%%%%%%%%%%%%%%%%%%%%%%%%
%\subsection{Conclusions}
%The Bland Altman Plot can be used to visually examine the bias and
%precision of two sets of measurements. It can also be used to
%determine whether there are any features present (such as the fan
%effect, proportional bias, outliers). An estimate for the
%inter-method bias can be determined and plotted. The second part
%of the methodology is the limits of agreement. A pre-specification
%of what would constitute an acceptable range for differences is
%recommended. Further to \citet{mantha}, this recommendation seems
%to be largely overlooked. Also it is recommended that the sample
%size be determined in advance.  There is no specific guidance in
%\citet{BA86} or \citet{BA99} in this regards. \citet{lin} notes
%that due attention has not been paid to sample size. Shewhart
%control limits, which are of similar construction to limits of
%agreement, are based on the process standard deviation, derived
%from at least 100 historical values. Furthermore there is a
%ambiguity as to what the limits of agreement are exactly, some
%authors regarding them as prediction intervals, other as tolerance
%intervals. Potentially some analysts may use the limits of
%agreement as if they were equivalent to Shewhart control charts.


%\sigma^{2}_{A} is estimated using $0.5 MSB = S_{AA}/(n - 1)$.
%\sigma^{2}_{D}is estimated using $2 MSE = S_{DD}/(n - l)$. $\rho =
%corr(A, D)$ is estimated from the regression of D on A. For a 95\%
%confidence ellipse, the chi square critical value is 5.99.





\subsubsection{Likelihood Ratio Tests} The problem with REML for
model building is that the "likelihoods" obtained for different
fixed effects are not comparable. Hence it is not valid to compare
models with different fixed effects using a likelihood ratio test
or AIC when REML is used to estimate the model. Therefore models
derived using ML must be used instead.




\end{document} 