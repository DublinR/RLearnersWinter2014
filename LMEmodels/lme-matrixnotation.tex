\documentclass[main.tex]{subfiles}
\begin{document}
\section{Matrix Formulation} There are matrix (i.e multivariate)
formulations of both fixed effects models and random effects
models. \citet{BrownPrescott} remarks that the matrix notation
makes the underlying theory of mixed effects models much easier to
work with. The fixed effects models can be specified as follows;

\begin{equation}
\textbf{Y} = \textbf{Xb} + \textbf{e}
\end{equation}

\textbf{Y} is the vector of $n$ observations, with dimension $n
\times 1$. \textbf{b} is a vector of fixed $p$ effects, and has
dimension $p \times 1$. It is composed of coefficients, with the
first element being the population mean. For the skin tumour
example, with the three specified fixed effects, $p=4$. \textbf{X}
is known as the design `matrix', model matrix for fixed effects,
and comprises $0$s or $1$s, depending on whether the relevant
fixed effects have any effect on the observation is question.
\textbf{X} has dimension $n \times p$. \textbf{e} is the vector of
residuals with dimension $n \times 1$.

The random effects models can be specified similarly. \textbf{Z}
is known as the `model matrix for random effects', and also
comprises $0$s or $1$s. It has dimension $n \times q$. \textbf{u}
is a vector of random $q$ effects, and has dimension $q \times 1$.

\begin{equation}
\textbf{Y} = \textbf{Zu} + \textbf{e}
\end{equation}

Again, once the component fixed effects and random effects
components are considered, progression to a mixed model
formulation is a simple step. Further to \citet{lw82}, it is
conventional to formulate a mixed effects model in matrix form as
follows:

\begin{equation}
\textbf{Y} = \textbf{Xb} + \textbf{Zu} + \textbf{e}
\end{equation}

($E(\textbf{u})=0$, $E(\textbf{e})=0 $ and $E(\textbf{y}) =
\textbf{Xb}$)

\section{Mixed Model Calculations}
\subsection{Formulation of the Variance Matrix V}
\textbf{V} , the variance matrix of \textbf{Y}, can be expressed
as follows;
\begin{eqnarray}
\textbf{V}= var ( \textbf{Xb} + \textbf{Zu} + \textbf{e})\\
\textbf{V}= var ( \textbf{Xb} ) + var (\textbf{Zu}) +
var(\textbf{e}))
\end{eqnarray}

%%%%%%%%%%%%%%%%%%%%%%%%%%%%%
$var(\textbf{Xb})$ is known to be zero. $var(\textbf{Zu})$ can be
written as $Zvar(\textbf{u})Z^{T}$. $Z$ is a matrix of constants.
By letting $var(u) = G$ (i.e $\textbf{u} ~ N(0,\textbf{G})$), this
becomes $ZGZ^{T}$. This specifies the covariance due to random
effects. The residual covariance matrix $var(e)$ is denoted as
$R$, ($\textbf{e} ~ N(0,\textbf{R})$). Residual are uncorrelated,
hence \textbf{R} is equivalent to $\sigma^{2}$\textbf{I}, where
\textbf{I} is the identity matrix. The variance matrix \textbf{V}
can therefore be written as;

\begin{equation}
\textbf{V}  = ZGZ^{T} + \textbf{R}
\end{equation}


\end{document}