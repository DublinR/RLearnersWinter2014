\documentclass[]{article}


\begin{document}

\section*{Ordinal Regression}

\subsection*{Important Considerations}

\begin{description}
\item[Perfect prediction:] Perfect prediction means that one value of a predictor variable is associated with only one value of the response variable. If this happens, Stata will usually issue a note at the top of the output and will drop the cases so that the model can run.
\item[Sample size:] Both ordered logistic and ordered probit, using maximum likelihood estimates, require sufficient sample size. How big is big is a topic of some debate, but they almost always require more cases than OLS regression.
\item[Empty cells or small cells:] You should check for empty or small cells by doing a crosstab between categorical predictors and the outcome variable. If a cell has very few cases, the model may become unstable or it might not run at all.
\item[Pseudo-R-squared:] There is no exact analog of the R-squared found in OLS. There are many versions of pseudo-R-squares. Please see Long and Freese 2005 for more details and explanations of various pseudo-R-squares.
\item[Diagnostics:] Doing diagnostics for non-linear models is difficult, and ordered logit/probit models are even more difficult than binary models.
\end{description}
\end{document}
