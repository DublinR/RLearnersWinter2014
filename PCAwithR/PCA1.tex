%- http://www.utstat.toronto.edu/~brunner/oldclass/appliedf12/source/
%--------------------------------------------------------------------------------------------%


Principal Components and Factor Analysis 

This section covers principal components and factor analysis. The later includes both exploratory and confirmatory methods.

Principal Components

The princomp( ) function produces an unrotated principal component analysis. 

# Pricipal Components Analysis
 # entering raw data and extracting PCs 
# from the correlation matrix 
fit <- princomp(mydata, cor=TRUE)
 summary(fit) # print variance accounted for 
loadings(fit) # pc loadings 
plot(fit,type="lines") # scree plot 
fit$scores # the principal components
 biplot(fit) 

%============================================================================ %

Use cor=FALSE to base the principal components on the covariance matrix. 
Use the covmat= option to enter a correlation or covariance matrix directly. 
If entering a covariance matrix, include the option n.obs=.


%============================================================================ %
The principal( ) function in the psych package can be used to extract and rotate principal components. 

# Varimax Rotated Principal Components
 # retaining 5 components 
library(psych)
 fit <- principal(mydata, nfactors=5, rotate="varimax")
 fit # print results 

mydata can be a raw data matrix or a covariance matrix. Pairwise deletion of missing data is used. rotate can "none", "varimax", "quatimax", "promax", "oblimin", "simplimax", or "cluster" . 

%============================================================================ %
\subsection*{Exploratory Factor Analysis}

The \texttt{factanal( )} function produces maximum likelihood factor analysis. 

# Maximum Likelihood Factor Analysis
 # entering raw data and extracting 3 factors, 
# with varimax rotation 
fit <- factanal(mydata, 3, rotation="varimax")
 print(fit, digits=2, cutoff=.3, sort=TRUE)
 # plot factor 1 by factor 2 
load <- fit$loadings[,1:2] 
plot(load,type="n") # set up plot 
text(load,labels=names(mydata),cex=.7) # add variable names 


The rotation= options include "varimax", "promax", and "none". Add the option scores="regression" or "Bartlett" to produce factor scores. Use the covmat= option to enter a correlation or covariance matrix directly. If entering a covariance matrix, include the option n.obs=.

The \texttt{factor.pa( )} function in the \textbf{\textit{psych}} package offers a number of factor analysis related functions, including principal axis factoring. 

# Principal Axis Factor Analysis
 library(psych)
 fit <- factor.pa(mydata, nfactors=3, rotation="varimax")
 fit # print results 

mydata can be a raw data matrix or a covariance matrix. Pairwise deletion of missing data is used. Rotation can be "varimax" or "promax". 

%========================================================================================================== %
\subsection{Determining the Number of Factors to Extract}

A crucial decision in exploratory factor analysis is how many factors to extract. The nFactors package offer a suite of functions to aid in this decision. Details on this methodology can be found in a PowerPoint presentation by Raiche, Riopel, and Blais. Of course, any factor solution must be interpretable to be useful. 
\begin{verbatim}

# Determine Number of Factors to Extract
 library(nFactors)
 ev <- eigen(cor(mydata)) # get eigenvalues
 ap <- parallel(subject=nrow(mydata),var=ncol(mydata),
   rep=100,cent=.05)
 nS <- nScree(x=ev$values, aparallel=ap$eigen$qevpea)
 plotnScree(nS) 
\end{verbatim}